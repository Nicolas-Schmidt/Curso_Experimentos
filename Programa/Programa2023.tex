% Options for packages loaded elsewhere
\PassOptionsToPackage{unicode}{hyperref}
\PassOptionsToPackage{hyphens}{url}
\PassOptionsToPackage{dvipsnames,svgnames*,x11names*}{xcolor}
%
\documentclass[
  12pt,
]{article}
\usepackage{lmodern}
\usepackage{amssymb,amsmath}
\usepackage{ifxetex,ifluatex}
\ifnum 0\ifxetex 1\fi\ifluatex 1\fi=0 % if pdftex
  \usepackage[T1]{fontenc}
  \usepackage[utf8]{inputenc}
  \usepackage{textcomp} % provide euro and other symbols
\else % if luatex or xetex
  \usepackage{unicode-math}
  \defaultfontfeatures{Scale=MatchLowercase}
  \defaultfontfeatures[\rmfamily]{Ligatures=TeX,Scale=1}
\fi
% Use upquote if available, for straight quotes in verbatim environments
\IfFileExists{upquote.sty}{\usepackage{upquote}}{}
\IfFileExists{microtype.sty}{% use microtype if available
  \usepackage[]{microtype}
  \UseMicrotypeSet[protrusion]{basicmath} % disable protrusion for tt fonts
}{}
\makeatletter
\@ifundefined{KOMAClassName}{% if non-KOMA class
  \IfFileExists{parskip.sty}{%
    \usepackage{parskip}
  }{% else
    \setlength{\parindent}{0pt}
    \setlength{\parskip}{6pt plus 2pt minus 1pt}}
}{% if KOMA class
  \KOMAoptions{parskip=half}}
\makeatother
\usepackage{xcolor}
\IfFileExists{xurl.sty}{\usepackage{xurl}}{} % add URL line breaks if available
\IfFileExists{bookmark.sty}{\usepackage{bookmark}}{\usepackage{hyperref}}
\hypersetup{
  pdftitle={Diseño e implementación de experimentos en ciencias sociales},
  pdfauthor={Maestría en Economía},
  colorlinks=true,
  linkcolor=Maroon,
  filecolor=Maroon,
  citecolor=Blue,
  urlcolor=blue,
  pdfcreator={LaTeX via pandoc}}
\urlstyle{same} % disable monospaced font for URLs
\usepackage[margin=1in]{geometry}
\usepackage{graphicx}
\makeatletter
\def\maxwidth{\ifdim\Gin@nat@width>\linewidth\linewidth\else\Gin@nat@width\fi}
\def\maxheight{\ifdim\Gin@nat@height>\textheight\textheight\else\Gin@nat@height\fi}
\makeatother
% Scale images if necessary, so that they will not overflow the page
% margins by default, and it is still possible to overwrite the defaults
% using explicit options in \includegraphics[width, height, ...]{}
\setkeys{Gin}{width=\maxwidth,height=\maxheight,keepaspectratio}
% Set default figure placement to htbp
\makeatletter
\def\fps@figure{htbp}
\makeatother
\setlength{\emergencystretch}{3em} % prevent overfull lines
\providecommand{\tightlist}{%
  \setlength{\itemsep}{0pt}\setlength{\parskip}{0pt}}
\setcounter{secnumdepth}{-\maxdimen} % remove section numbering
\usepackage{setspace}
\singlespacing

\title{Diseño e implementación de experimentos en ciencias sociales}
\author{Maestría en Economía}
\date{\emph{Año 2023}}

\begin{document}
\maketitle

\textbf{Docentes}:

Santiago López Cariboni \newline e-mail:
\href{mailto:santiago.lopez@cienciassociales.edu.uy}{\nolinkurl{santiago.lopez@cienciassociales.edu.uy}}
\newline Personal website: \url{https://www.lopez-cariboni.info/}

Luciana Cantera \newline email:
\href{mailto:luciana.cantera@cienciassociales.edu.uy}{\nolinkurl{luciana.cantera@cienciassociales.edu.uy}}

Nicolás Schmidt \newline email:
\href{mailto:nschmidt@cienciassociales.edu.uy}{\nolinkurl{nschmidt@cienciassociales.edu.uy}}

Lucía Suárez \newline e-mail:
\href{mailto:lucia.suarez@cienciassociales.edu.uy}{\nolinkurl{lucia.suarez@cienciassociales.edu.uy}}

\textbf{Créditos}: 6 \newline \textbf{Carga horaria}: 30 horas \newline
\textbf{Modalidad de enseñanza}: Teórico y taller

\hypertarget{descripciuxf3n}{%
\subsubsection{Descripción:}\label{descripciuxf3n}}

Este curso abarca el diseño, la realización y el análisis de
experimentos en ciencias sociales. Revisa los experimentos de campo,
experimentos naturales, de encuesta y de laboratorio.

Los materiales del curso se encuentran en la siguiente página web, donde
se irán actualizando con el correr de las clases:

{[}web del curso{]}

\hypertarget{conocimientos-previos-recomendados}{%
\subsubsection{Conocimientos previos
recomendados:}\label{conocimientos-previos-recomendados}}

Este curso presupone una familiaridad con la estadística al nivel de
grado (regresión para las Ciencias Sociales), que puede alcanzarse a
través de cursos en algunas diversas disciplinas.

Los estudiantes que no tomaron métodos en ciencias sociales pueden
ponerse en contacto con el equipo docente antes de inscribirse en el
curso para discutir su preparación.

El curso asume una familiaridad básica con el entorno estadístico R e
implica un uso sustancial de R en la mayoría de las sesiones de clase.
Los estudiantes sin conocimientos previos de R deben contactar al equipo
docente para adquirir de forma guiada esos conocimientos durante la
semana previa al inicio del curso.

\hypertarget{objetivos-de-aprendizaje}{%
\subsubsection{Objetivos de
aprendizaje:}\label{objetivos-de-aprendizaje}}

\begin{itemize}
\tightlist
\item
  Cómo identificar y abordar las principales amenazas para los diseños
  experimentales
\item
  Cómo implementar componentes clave del diseño y análisis experimental
  en código
\item
  Cómo evaluar las opciones de diseño en sus propios experimentos a
  través de la simulación
\item
  Adquirir experiencia replicando el diseño y análisis de experimentos
  publicados destacados
\item
  Cómo diseñar un experimento y preparar un pre-registro completo.
\end{itemize}

\hypertarget{muxe9todo-de-trabajo}{%
\subsubsection{Método de trabajo:}\label{muxe9todo-de-trabajo}}

Se realizará en la modalidad de clases teóricas y taller. Los
estudiantes deben realizar lecturas previas a cada clase y trabajar de
manera individual o en grupo durante los talleres.

\hypertarget{sistema-de-evaluaciuxf3n}{%
\subsubsection{Sistema de evaluación}\label{sistema-de-evaluaciuxf3n}}

El curso se evalúa en base a un examen final. El examen consiste en un
paper que involucra el diseño completo de un experimento. Adquieren
derecho a realizar el examen aquellos alumnos que a) hayan hecho entrega
de los trabajos prácticos (talleres) que realizan durante el curso y b)
hayan asistido al mínimo obligatorio de clases según el reglamento de la
Maestría en Economía.

\hypertarget{bibliografuxeda-buxe1sica-por-orden-de-importancia-para-el-curso}{%
\subsubsection{Bibliografía básica (por orden de importancia para el
curso):}\label{bibliografuxeda-buxe1sica-por-orden-de-importancia-para-el-curso}}

Gerber, Alan S., and Donald P. Green. 2012. \emph{Field Experiments:
Design, Analysis, and Interpretation.} New York: W.W. Norton. (FEDAI).

Imbens, G. W., \& Rubin, D. B. (2015). Causal inference in statistics,
social, and biomedical sciences. Cambridge University Press. (CISSBS)

Athey, Susan, and Guido W Imbens. 2017. ``The Econometrics of Randomized
Experiments.'' In Handbook of Economic Field Experiments, Elsevier,
73--140. (ERE)

Blair, Graeme, Jasper Cooper, Alexander Coppock, y Macartan Humphreys.
\emph{Research Design in the Social Sciences Declaration, Diagnosis, and
Redesign}. Forthcoming, Princeton University Press. (R3DR). Disponible
online en: \url{https://book.declaredesign.org/}

Glennerster, Rachel, and Kudzai Takavarasha. 2013. \emph{Running
Randomized Evaluations: A Practical Guide}. Princeton: Princeton
UP.(RRE)

Jake Bowers, Maarten Voors, and Nahomi Ichino. Traducido por Lily
Medina,
\href{https://lilymedina.github.io/theory_and_practice_of_field_experiments/}{La
teoría y la práctica de los experimentos de campo: Una introducción de
los Learning Days de EGAP}

\hypertarget{recursos-computacionales}{%
\subsubsection{Recursos
computacionales:}\label{recursos-computacionales}}

\begin{itemize}
\tightlist
\item
  Grolemund, Garrett and Hadley Wickham.
  \href{https://r4ds.hadley.nz/}{R for Data Science (2e)}
\item
  \href{https://sicss.io/boot_camp/}{Computational Social Science
  Bootcamp}
\item
  \href{https://declaredesign.org/getting-started/}{DeclareDesign}
  software
\item
  \href{https://posit.co/resources/cheatsheets/}{\texttt{tidyverse}
  cheat sheets}
\item
  Compilado de tutoriales de
  \href{https://github.com/Chris-Engelhardt/data_sci_guide/}{\texttt{R}
  y otros recursos}
\item
  \href{https://posit.cloud/learn/primers}{RStudio \texttt{R} primers}
\item
  Paquetes de \texttt{R} útiles para experimentalistas \texttt{R}:

  \begin{itemize}
  \tightlist
  \item
    \texttt{randomizr}: facilita esquemas de aleatorización comunes
  \item
    \texttt{estimatr}: estimadores para experimentos basados en diseño
  \item
    \texttt{blockTools}: construir bloques/estratos para experimentos
    aleatorios por bloques
  \item
    \texttt{ri2}: implementa inferencia de aleatorización
  \item
  \item
    \texttt{GRF}: \href{https://grf-labs.github.io/grf/}{generalized
    random forests}
  \item
    rlearner: \href{https://github.com/xnie/rlearner}{R-learner for
    Quasi-Oracle Estimation of Heterogeneous Treatment Effects}".
  \end{itemize}
\item
  \href{https://www.otree.org/}{\texttt{oTree}}. Plataforma para
  programar expeirmentos de laboratorio.
\item
  \href{https://www.qualtrics.com/}{\texttt{Qualtrics}}. Plataforma para
  programar expeirmentos de encuesta.
\end{itemize}

\hypertarget{contenidos}{%
\subsection{Contenidos}\label{contenidos}}

\hypertarget{tema-1.-inferencia-causal-1-clase}{%
\subsubsection{Tema 1. Inferencia causal (1
clase)}\label{tema-1.-inferencia-causal-1-clase}}

\begin{itemize}
\tightlist
\item
  \emph{El modelo de causalidad Neyman--Rubin}
\item
  \emph{Experimentos aleatorizados y validez}
\item
  \emph{Asignación aleatoria simple}
\item
  \emph{Estimandos (ATE, ITT, CACE, SATE, PATE, ATT, CATE, mediación)}
\end{itemize}

\hypertarget{tema-2.-anuxe1lisis-bajo-asignaciuxf3n-aleatoria-simple-1-clase}{%
\subsubsection{Tema 2. Análisis bajo asignación aleatoria simple (1
clase)}\label{tema-2.-anuxe1lisis-bajo-asignaciuxf3n-aleatoria-simple-1-clase}}

\begin{itemize}
\tightlist
\item
  \emph{Inferencia de aleatorización (p-valores exactos para hipótesis
  nulas nítidas)}
\item
  \emph{Regresión y ajuste por covariables}
\end{itemize}

\hypertarget{tema-3.-estrategias-de-aleatorizaciuxf3n-y-anuxe1lisis-basado-en-diseuxf1o-2-clases}{%
\subsubsection{Tema 3. Estrategias de aleatorización y análisis basado
en diseño (2
clases)}\label{tema-3.-estrategias-de-aleatorizaciuxf3n-y-anuxe1lisis-basado-en-diseuxf1o-2-clases}}

\begin{itemize}
\tightlist
\item
  \emph{Poder estadístico. Fórmula analítica y simulaciones.}
\item
  \emph{Uso de covariables en el diseño. Asignación por bloques y
  análisis}
\item
  \emph{Asignación aleatoria por clusters y análisis}
\end{itemize}

\hypertarget{tema-4.-diseuxf1os-experimentales-4-clases}{%
\subsubsection{Tema 4. Diseños experimentales (4
clases)}\label{tema-4.-diseuxf1os-experimentales-4-clases}}

\begin{itemize}
\tightlist
\item
  \emph{Diseños de aliento y cumplimiento imperfecto}
\item
  \emph{Diseños multi-rama}
\item
  \emph{Diseños factoriales}
\item
  \emph{Diseños de mediación}
\item
  \emph{Diseños para estimar efectos de derrame}
\item
  \emph{Diseños adaptativos}
\item
  \emph{Diagnóstico de diseños}
\end{itemize}

\hypertarget{tema-5.-anuxe1lisis-estaduxedstico-de-experimentos-3-clases}{%
\subsubsection{Tema 5. Análisis estadístico de experimentos (3
clases)}\label{tema-5.-anuxe1lisis-estaduxedstico-de-experimentos-3-clases}}

\begin{itemize}
\tightlist
\item
  \emph{Análisis de efectos heterogéneos y variables pre-tratamiento}
\item
  \emph{Corrección por hipótesis múltiples}
\item
  \emph{Spillovers (contaminación)}
\item
  \emph{Non-compliance}
\item
  \emph{Atrición}
\end{itemize}

\hypertarget{tema-6.-tipos-de-experimentos-en-ciencias-sociales-3-clases}{%
\subsubsection{Tema 6. Tipos de experimentos en ciencias sociales (3
clases)}\label{tema-6.-tipos-de-experimentos-en-ciencias-sociales-3-clases}}

\begin{itemize}
\tightlist
\item
  \emph{Experimentos de campo}
\item
  \emph{Experimentos naturales}
\item
  \emph{Experimentos de encuesta}
\item
  \emph{Experimentos de laboratorio}
\end{itemize}

\hypertarget{tema-7.-pre-registro-de-experimentos-1-clase}{%
\subsubsection{Tema 7. Pre-registro de experimentos (1
clase)}\label{tema-7.-pre-registro-de-experimentos-1-clase}}

\begin{itemize}
\tightlist
\item
  \emph{Elaboración de un plan de análisis}
\item
  \emph{Ética en la investigación experimental}
\item
  \emph{Materiales de pre-registro}
\end{itemize}

\hypertarget{lecturas-por-tema}{%
\subsection{Lecturas por tema}\label{lecturas-por-tema}}

\hypertarget{tema-1.-inferencia-causal}{%
\subsubsection{Tema 1. Inferencia
causal}\label{tema-1.-inferencia-causal}}

\begin{itemize}
\tightlist
\item
  \emph{El modelo de causalidad Neyman--Rubin}
\item
  \emph{Experimentos aleatorizados y validez}
\item
  \emph{Asignación aleatoria simple}
\item
  \emph{Estimandos (ATE, ITT, CACE, SATE, PATE, ATT, CATE, mediación)}
\end{itemize}

FEDAI, capítulos 1-2

CISSBS, capítulos 1-2.

\href{https://egap.org/resource/10-types-of-treatment-effect-you-should-know-about/}{10
Types of Treatment Effect You Should Know About}

Aronow, P. M. and Samii, C. (2016). ``Does regression produce
representative estimates of causal effects?'' \emph{American Journal of
Political Science}, 60(1):250--267

Barabas, J. and Jerit, J. (2010). ``Are survey experiments externally
valid?'' \emph{American Political Science Review}, 104(2):226--242

\hypertarget{tema-2.-anuxe1lisis-bajo-asignaciuxf3n-aleatoria-simple}{%
\subsubsection{Tema 2. Análisis bajo asignación aleatoria
simple}\label{tema-2.-anuxe1lisis-bajo-asignaciuxf3n-aleatoria-simple}}

\begin{itemize}
\tightlist
\item
  \emph{Inferencia de aleatorización (p-valores exactos para hipótesis
  nulas nítidas)}
\end{itemize}

FEDAI, capítulo 3

CISSBS capítulos 5 y 6, capítulo 9 (secciones 9.3 y 9.8), capítulo 10
(sección 10.3)

Duflo, E., Glennerster, R., \& Kremer, M. (2007). Using randomization in
development economics research: A toolkit. Handbook of development
economics, 4, 3895-3962. (capítulos 4 y 7)

EGAP,
\href{https://egap.org/resource/10-things-to-know-about-hypothesis-testing/}{10
Things to Know About Randomization Inference}

Peng Ding, Avi Feller, and Luke Miratrix (2016), ``Randomization
Inference for Treatment Effect Variation,'' Journal of the Royal
Statistical Society, Series B 78: 655--671

\begin{itemize}
\tightlist
\item
  \emph{Regresión y ajuste por covariables}
\end{itemize}

FEDAI, secciones 4.1, 4.2, y 4.3.

Freedman, David A. 2008. ``On Regression Adjustments in Experiments with
Several Treatments.'' \emph{Annals of Applied Statistics} 2(1): 176--96.

Lin, Winston. 2013. ``Agnostic Notes on Regression Adjustments to
Experimental Data: Reexamining Freedman's Critique.'' \emph{The Annals
of Applied Statistics} 7(1): 295--318.

Wager, Stefan, Wenfei Du, Jonathan Taylor, and Robert J. Tibshirani.
2016. ``High-Dimensional Regression Adjustments in Randomized
Experiments.'' \emph{Proceedings of the National Academy of Sciences of
the United States of America} 113(45): 12673--78.

\hypertarget{tema-3.-estrategias-de-aleatorizaciuxf3n-y-anuxe1lisis-basado-en-diseuxf1o}{%
\subsubsection{Tema 3. Estrategias de aleatorización y análisis basado
en
diseño}\label{tema-3.-estrategias-de-aleatorizaciuxf3n-y-anuxe1lisis-basado-en-diseuxf1o}}

\begin{itemize}
\tightlist
\item
  \emph{Poder estadístico. Fórmula analítica y simulaciones.}
\end{itemize}

EGAP,
\href{https://egap.org/resource/10-things-to-know-about-statistical-power/}{10
Things to Know About Statistical Power}

\begin{itemize}
\tightlist
\item
  \emph{Uso de covariables en el diseño. Asignación por bloques y
  análisis}
\end{itemize}

FEDAI, capítulos 3.6.1, 4.4, y 4.5.

EGAP,
\href{https://egap.org/resource/10-things-to-know-about-multisite-or-block-randomized-trials/}{10
Things to Know About Multisite or Block-Randomized Trials}

Moore, Ryan T. 2012. ``Multivariate Continuous Blocking to Improve
Political Science Experiments.'' \emph{Political Analysis} 20(4):
460--79.

Moore, Ryan T., and Sally A. Moore. 2013. ``Blocking for Sequential
Political Experiments.'' \emph{Political Analysis} 21(4): 507--23.

Coppock, Alexander. 2019. ``Randomizr: Easy-to-Use Tools for Common
Forms of Random Assignment and Sampling.''

Moore, Ryan T., and Keith Schnakenberg. 2016. ``BlockTools: Blocking,
Assignment, and Diagnosing Interference in Randomized Experiments.''

\begin{itemize}
\tightlist
\item
  \emph{Asignación aleatoria por clusters y análisis}
\end{itemize}

FEDAI, capítulo 3.6.2.

EGAP,
\href{https://egap.org/resource/10-things-to-know-about-cluster-randomization/}{10
Things to Know About Cluster Randomization}

\hypertarget{tema-4.-diseuxf1os-experimentales}{%
\subsubsection{Tema 4. Diseños
experimentales}\label{tema-4.-diseuxf1os-experimentales}}

\begin{itemize}
\tightlist
\item
  \emph{Diseños de aliento y cumplimiento imperfecto}
\end{itemize}

FEDAI, capítulos 5-6.

Athey, Susan, and Guido W Imbens. 2017. ``The Econometrics of Randomized
Experiments.'' In Handbook of Economic Field Experiments, Elsevier,
73--140.

\begin{itemize}
\tightlist
\item
  \emph{Diseños multi-rama}
\end{itemize}

Gerber, Alan S, and Donald P Green. 2012. \emph{Field Experiments:
Design, Analysis, and Interpretation}. WW Norton.

Athey, Susan, and Guido W Imbens. 2017. ``The Econometrics of Randomized
Experiments.'' In Handbook of Economic Field Experiments, Elsevier,
73--140.

\begin{itemize}
\tightlist
\item
  \emph{Diseños factoriales}
\end{itemize}

Egami, Naoki, and Kosuke Imai. 2018. ``Causal Interaction in Factorial
Experiments: Application to Conjoint Analysis.'' \emph{Journal of the
American Statistical Association}: 1--34.

Tom Pepinsky,
\href{https://tompepinsky.com/2022/10/19/factorial-experiments-conjoints-ames-and-amces/}{Factorial
Experiments, Conjoints, AMEs, and AMCEs}

\begin{itemize}
\tightlist
\item
  \emph{Diseños de mediación}
\end{itemize}

FEDAI, capítulo 10.

Imai, Kosuke, Luke Keele, Dustin Tingley, and Teppei Yamamoto. 2011.
``Unpacking the Black Box of Causality: Learning about Causal Mechanisms
from Experimental and Observational Studies.'' \emph{American Political
Science Review} 105(4): 765--89.

Imai, Kosuke, Dustin Tingley, and Teppei Yamamoto. 2013. ``Experimental
Designs for Identifying Causal Mechanisms.'' \emph{Journal of the Royal
Statistical Society. Series A: Statistics in Society} 176(1): 5--51.

Imai, Kosuke, and Teppei Yamamoto. 2013. ``Identification and
Sensitivity Analysis for Multiple Causal Mechanisms: Revisiting Evidence
from Framing Experiments.'' \emph{Political Analysis} 21(02): 141--71.

Tingley, Dustin et al.~2014. ``Mediation: R Package for Causal Mediation
Analysis.'' Journal of Statistical Software 59(5): 1--38.

\begin{itemize}
\tightlist
\item
  \emph{Diseños para estimar efectos de derrame}
\end{itemize}

Nickerson, D. W. (2008). ``Is voting contagious? Evidence from two field
experiments.'' \emph{American political Science Review}, 49-57.

Sinclair, B., McConnell, M., \& Green, D. P. (2012). ``Detecting
spillover effects: Design and analysis of multilevel experiments.''
\emph{American Journal of Political Science}, 56(4), 1055-1069.

\begin{itemize}
\tightlist
\item
  \emph{Diseños adaptativos}
\end{itemize}

Offer-westort, Molly, and Alexander Coppock. 2018. ``Adaptive
Experimental Design : Prospects and Applications in Political Science
Prepared for Presentation at the Annual Meeting of the American
Political.''

\begin{itemize}
\tightlist
\item
  \emph{Diagnóstico de diseños y potencia estadística}
\end{itemize}

Blair, Graeme, Jasper Cooper, Alexander Coppock, and Macartan Humphreys.
2019. ``Declaring and Diagnosing Research Designs.'' \emph{American
Political Science Review} 113(3): 838--59.

\hypertarget{tema-5.-anuxe1lisis-estaduxedstico-de-experimentos}{%
\subsubsection{Tema 5. Análisis estadístico de
experimentos}\label{tema-5.-anuxe1lisis-estaduxedstico-de-experimentos}}

\begin{itemize}
\tightlist
\item
  \emph{Análisis de efectos heterogéneos y variables pre-tratamiento}
\end{itemize}

FEDAI, capítulo 9.

EGAP,
\href{https://egap.org/resource/10-things-to-know-about-heterogeneous-treatment-effects/}{10
Things to Know About Heterogeneous Treatment Effects}

Sesgo post-tratamiento:

Montgomery, Jacob M., Brendan Nyhan, and Michelle Torres. 2018. ``How
Conditioning on Posttreatment Variables Can Ruin Your Experiment and
What to Do about It.'' \emph{American Journal of Political Science}
62(3): 760--75.

Variables pre-tratamiento:

Carneiro, Pedro, Sokbae Lee, and Daniel Wilhelm. 2020. ``Optimal Data
Collection for Randomized Control Trials.'' \emph{Econometrics Journal}
23(1): 1--31.

Heterogeneidad con aprendizaje automático y alta dimensionalidad:

Egami, Naoki, and Kosuke Imai. 2018. ``Causal Interaction in Factorial
Experiments: Application to Conjoint Analysis.'' \emph{Journal of the
American Statistical Association}: 1--34.

Athey, Susan, and Guido W. Imbens. 2019. ``Machine Learning Methods That
Economists Should Know About.'' \emph{Annual Review of Economics} 11(1):
685--725.

Athey, Susan, Julie Tibshirani, and Stefan Wager. 2019. ``Generalized
Random Forests.'' \emph{Annals of Statistics} 47(2): 1179--1203.

Wager, Stefan, and Susan Athey. 2018. ``Estimation and Inference of
Heterogeneous Treatment Effects Using Random Forests.'' \emph{Journal of
the American Statistical Association} 113(523): 1228--42.

Athey, Susan, and Stefan Wager. 2019. ``Estimating Treatment Effects
with Causal Forests: An Application.'' \emph{Observational Studies}
5(2): 37--51.

Xinkun Nie and Stefan Wager. 2021. Quasi-Oracle Estimation of
Heterogeneous Treatment Effects. \emph{Biometrika}, 108(2).

\begin{itemize}
\tightlist
\item
  \emph{Corrección por hipótesis múltiples}
\end{itemize}

EGAP,
\href{https://egap.org/resource/10-things-to-know-about-multiple-comparisons/}{10
Things to Know About Multiple Comparisons}

\hypertarget{tema-6.-tipos-de-experimentos-en-ciencias-sociales}{%
\subsubsection{Tema 6. Tipos de experimentos en ciencias
sociales}\label{tema-6.-tipos-de-experimentos-en-ciencias-sociales}}

\begin{itemize}
\tightlist
\item
  \emph{Experimentos naturales}
\end{itemize}

Dunning, Thad. 2012. \emph{Natural Experiments in the Social Sciences
Natural Experiments in the Social Sciences}.

Dunning, Thad. 2008. ``Improving Causal Inference: Strengths and
Limitations of Natural Experiments.'' \emph{Political Research
Quarterly} 61(2): 282--293.

\begin{itemize}
\tightlist
\item
  \emph{Experimentos de campo}
\end{itemize}

Gerber, Alan S, and Donald P Green. 2012. \emph{Field Experiments:
Design, Analysis, and Interpretation}. WW Norton.

Coppock, A. and Green, D. P. (2015). ``Assessing the correspondence
between experimental results obtained in the lab and field: A review of
recent social science research''. \emph{Political Science Research and
Methods}, 3(1):113--131

List, J. A. (2011). ``Why economists should conduct field experiments
and 14 tips for pulling one off''. \emph{The Journal of Economic
Perspectives}, 25(3):3--15

\begin{itemize}
\tightlist
\item
  \emph{Experimentos de encuesta}
\end{itemize}

Generales:

Gaines, Brian J., James H. Kuklinski, and Paul J. Quirk. 2007. ``The
Logic of the Survey Experiment Reexamined.'' \emph{Political Analysis}
15(01): 1--20.

Coppock, Alexander. 2018. ``Generalizing from Survey Experiments
Conducted on Mechanical Turk: A Replication Approach.'' \emph{Political
Science Research and Methods} (2015): 1--16.

Tobergte, David R., and Shirley Curtis. 2013. ``The Generalizability of
Survey Experiments.'' \emph{Journal of Chemical Information and
Modeling} 53(9): 1689--99.

Tratamientos de información

Linos, Katerina, and Kimberly Twist. 2018. ``Diverse Pre-Treatment
Effects in Survey Experiments.'' \emph{Journal of Experimental Political
Science} 5(2): 148--58.

Fernández-Albertos, José, and Alexander Kuo. 2018. ``Income Perception,
Information, and Progressive Taxation: Evidence from a Survey
Experiment.'' \emph{Political Science Research and Methods} 6(01):
83--110.

Cruces, Guillermo, Ricardo Perez-Truglia, and Martin Tetaz. 2013.
``Biased Perceptions of Income Distribution and Preferences for
Redistribution: Evidence from a Survey Experiment.'' \emph{Journal of
Public Economics} 98: 100--112.

Stokes, Leah C., and Christopher Warshaw. 2017. ``Renewable Energy
Policy Design and Framing Influence Public Support in the United
States.'' Nature Energy 2: 17107.

Conjoint:

Hainmueller, Jens, Daniel J. Hopkins, and Teppei Yamamoto. 2014.
``Causal Inference in Conjoint Analysis: Understanding Multidimensional
Choices via Stated Preference Experiments.'' \emph{Political Analysis}
22(1): 1--30.

Leeper, Thomas J., Sara B. Hobolt, and James Tilley. 2020. ``Measuring
Subgroup Preferences in Conjoint Experiments.'' \emph{Political
Analysis} 28(2): 207--21.

Experimentos de lista:

Blair, Graeme, and Kosuke Imai. 2012. ``Statistical Analysis of List
Experiments.'' \emph{Political Analysis} 20(1): 47--77.

Mecanismos causales:

Acharya, Avidit, Matthew Blackwell, and Maya Sen.~2018. ``Analyzing
Causal Mechanisms in Survey Experiments.'' \emph{Political Analysis}
26(4): 1--31.

Cumplimiento a la asignación y atención:

Aronow, Peter M., Jonathon Baron, and Lauren Pinson. 2019. ``A Note on
Dropping Experimental Subjects Who Fail a Manipulation Check.''
\emph{Political Analysis} 27(4): 572--89.

\begin{itemize}
\tightlist
\item
  \emph{Experimentos de laboratorio}
\end{itemize}

Levitt, Steven, D., and John A. List. 2007. ``What Do Laboratory
Experiments Measuring Social Preferences Reveal About the Real World?''
Journal of Economic Perspectives, 21 (2): 153-174.

Falk, A. and Heckman, J. J. (2009). ``Lab experiments are a major source
of knowledge in the social sciences''. \emph{Science},
326(5952):535--538

Habyarimana, J., Humphreys, M., Posner, D. N., and Weinstein, J. M.
(2007). ``Why does ethnic diversity undermine public goods provision?''
\emph{American Political Science Review}, 101(4):709--725

Oxley, D. R., Smith, K. B., Alford, J. R., Hibbing, M. V., Miller, J.
L., Scalora, M., Hatemi, P. K., and Hibbing, J. R. (2008). ``Political
attitudes vary with physiological traits''. \emph{Science},
321(5896):1667--1670

Levine, D. K. and Palfrey, T. R. (2007). The paradox of voter
participation? a laboratory study. American political science Review,
101(1):143--158

\hypertarget{tema-7.-pre-registro-de-experimentos}{%
\subsubsection{Tema 7. Pre-registro de
experimentos}\label{tema-7.-pre-registro-de-experimentos}}

\begin{itemize}
\tightlist
\item
  \emph{Elaboración de un plan de análisis}
\item
  \emph{Ética en la investigación experimental}
\item
  \emph{Materiales de pre-registro}
\end{itemize}

Bowers, J., \& Testa, P. F. (2019). ``Better Government, Better Science:
The Promise of and Challenges Facing the Evidence-Informed Policy
Movement.'' \emph{Annual Review of Political Science}, 22, 521-542.

\hypertarget{section}{%
\subsection{=======}\label{section}}

title: ``Diseño e implementación de experimentos en ciencias sociales''
author: - ``Maestría en Economía'' date: ``\emph{Año 2023}'' output:
pdf\_document header-includes: -

\usepackage{setspace}

\begin{itemize}
\item
  \singlespacing

  fontsize: 12pt urlcolor: blue ---
\end{itemize}

\textbf{Docentes}:

Santiago López Cariboni \newline e-mail:
\href{mailto:santiago.lopez@cienciassociales.edu.uy}{\nolinkurl{santiago.lopez@cienciassociales.edu.uy}}
\newline Personal website: \url{https://www.lopez-cariboni.info/}

Luciana Cantera \newline Nicolás Schmidt \newline

Lucía Suárez \newline e-mail:
\href{mailto:lucia.suarez@cienciassociales.edu.uy}{\nolinkurl{lucia.suarez@cienciassociales.edu.uy}}

\textbf{Créditos}: 6 \newline \textbf{Carga horaria}: 30 horas \newline
\textbf{Modalidad de enseñanza}: Teórico y taller

\hypertarget{descripciuxf3n-1}{%
\subsubsection{Descripción:}\label{descripciuxf3n-1}}

Este curso abarca el diseño, la realización y el análisis de
experimentos en ciencias sociales. Revisa los experimentos de campo,
experimentos naturales, de encuesta y de laboratorio.

Los materiales del curso se encuentran en la siguiente página web, donde
se irán actualizando con el correr de las clases:

{[}web del curso{]}

\hypertarget{conocimientos-previos-recomendados-1}{%
\subsubsection{Conocimientos previos
recomendados:}\label{conocimientos-previos-recomendados-1}}

Este curso presupone una familiaridad con la estadística al nivel de
grado (regresión para las Ciencias Sociales), que puede alcanzarse a
través de cursos en algunas diversas disciplinas.

Los estudiantes que no tomaron métodos en ciencias sociales pueden
ponerse en contacto con el equipo docente antes de inscribirse en el
curso para discutir su preparación.

El curso asume una familiaridad básica con el entorno estadístico R e
implica un uso sustancial de R en la mayoría de las sesiones de clase.
Los estudiantes sin conocimientos previos de R deben contactar al equipo
docente para adquirir de forma guiada esos conocimientos durante la
semana previa al inicio del curso.

\hypertarget{objetivos-de-aprendizaje-1}{%
\subsubsection{Objetivos de
aprendizaje:}\label{objetivos-de-aprendizaje-1}}

\begin{itemize}
\tightlist
\item
  Cómo identificar y abordar las principales amenazas para los diseños
  experimentales
\item
  Cómo implementar componentes clave del diseño y análisis experimental
  en código
\item
  Cómo evaluar las opciones de diseño en sus propios experimentos a
  través de la simulación
\item
  Adquirir experiencia replicando el diseño y análisis de experimentos
  publicados destacados
\item
  Cómo diseñar un experimento y preparar un pre-registro completo.
\end{itemize}

\hypertarget{muxe9todo-de-trabajo-1}{%
\subsubsection{Método de trabajo:}\label{muxe9todo-de-trabajo-1}}

Se realizará en la modalidad de clases teóricas y taller. Los
estudiantes deben realizar lecturas previas a cada clase y trabajar de
manera individual o en grupo durante los talleres.

\hypertarget{sistema-de-evaluaciuxf3n-1}{%
\subsubsection{Sistema de evaluación}\label{sistema-de-evaluaciuxf3n-1}}

El curso se evalúa en base a un examen final. El examen consiste en un
paper que involucra el diseño completo de un experimento. Adquieren
derecho a realizar el examen aquellos alumnos que a) hayan hecho entrega
de los trabajos prácticos (talleres) que realizan durante el curso y b)
hayan asistido al mínimo obligatorio de clases según el reglamento de la
Maestría en Economía.

\hypertarget{bibliografuxeda-buxe1sica-por-orden-de-importancia-para-el-curso-1}{%
\subsubsection{Bibliografía básica (por orden de importancia para el
curso):}\label{bibliografuxeda-buxe1sica-por-orden-de-importancia-para-el-curso-1}}

Gerber, Alan S., and Donald P. Green. 2012. \emph{Field Experiments:
Design, Analysis, and Interpretation.} New York: W.W. Norton. (FEDAI).

Imbens, G. W., \& Rubin, D. B. (2015). Causal inference in statistics,
social, and biomedical sciences. Cambridge University Press. (CISSBS)

Athey, Susan, and Guido W Imbens. 2017. ``The Econometrics of Randomized
Experiments.'' In Handbook of Economic Field Experiments, Elsevier,
73--140. (ERE)

Blair, Graeme, Jasper Cooper, Alexander Coppock, y Macartan Humphreys.
\emph{Research Design in the Social Sciences Declaration, Diagnosis, and
Redesign}. Forthcoming, Princeton University Press. (R3DR). Disponible
online en: \url{https://book.declaredesign.org/}

Glennerster, Rachel, and Kudzai Takavarasha. 2013. \emph{Running
Randomized Evaluations: A Practical Guide}. Princeton: Princeton
UP.(RRE)

Jake Bowers, Maarten Voors, and Nahomi Ichino. Traducido por Lily
Medina,
\href{https://lilymedina.github.io/theory_and_practice_of_field_experiments/}{La
teoría y la práctica de los experimentos de campo: Una introducción de
los Learning Days de EGAP}

\hypertarget{recursos-computacionales-1}{%
\subsubsection{Recursos
computacionales:}\label{recursos-computacionales-1}}

\begin{itemize}
\tightlist
\item
  Grolemund, Garrett and Hadley Wickham.
  \href{https://r4ds.hadley.nz/}{R for Data Science (2e)}
\item
  \href{https://sicss.io/boot_camp/}{Computational Social Science
  Bootcamp}
\item
  \href{https://declaredesign.org/getting-started/}{DeclareDesign}
  software
\item
  \href{https://posit.co/resources/cheatsheets/}{\texttt{tidyverse}
  cheat sheets}
\item
  Compilado de tutoriales de
  \href{https://github.com/Chris-Engelhardt/data_sci_guide/}{\texttt{R}
  y otros recursos}
\item
  \href{https://posit.cloud/learn/primers}{RStudio \texttt{R} primers}
\item
  Paquetes de \texttt{R} útiles para experimentalistas \texttt{R}:

  \begin{itemize}
  \tightlist
  \item
    \texttt{randomizr}: facilita esquemas de aleatorización comunes
  \item
    \texttt{estimatr}: estimadores para experimentos basados en diseño
  \item
    \texttt{blockTools}: construir bloques/estratos para experimentos
    aleatorios por bloques
  \item
    \texttt{ri2}: implementa inferencia de aleatorización
  \item
  \item
    \texttt{GRF}: \href{https://grf-labs.github.io/grf/}{generalized
    random forests}
  \item
    rlearner: \href{https://github.com/xnie/rlearner}{R-learner for
    Quasi-Oracle Estimation of Heterogeneous Treatment Effects}".
  \end{itemize}
\item
  \href{https://www.otree.org/}{\texttt{oTree}}. Plataforma para
  programar expeirmentos de laboratorio.
\item
  \href{https://www.qualtrics.com/}{\texttt{Qualtrics}}. Plataforma para
  programar expeirmentos de encuesta.
\end{itemize}

\hypertarget{contenidos-1}{%
\subsection{Contenidos}\label{contenidos-1}}

\hypertarget{tema-1.-inferencia-causal-1-clase-1}{%
\subsubsection{Tema 1. Inferencia causal (1
clase)}\label{tema-1.-inferencia-causal-1-clase-1}}

\begin{itemize}
\tightlist
\item
  \emph{El modelo de causalidad Neyman--Rubin}
\item
  \emph{Experimentos aleatorizados y validez}
\item
  \emph{Asignación aleatoria simple}
\item
  \emph{Estimandos (ATE, ITT, CACE, SATE, PATE, ATT, CATE, mediación)}
\end{itemize}

\hypertarget{tema-2.-anuxe1lisis-bajo-asignaciuxf3n-aleatoria-simple-1-clase-1}{%
\subsubsection{Tema 2. Análisis bajo asignación aleatoria simple (1
clase)}\label{tema-2.-anuxe1lisis-bajo-asignaciuxf3n-aleatoria-simple-1-clase-1}}

\begin{itemize}
\tightlist
\item
  \emph{Inferencia de aleatorización (p-valores exactos para hipótesis
  nulas nítidas)}
\item
  \emph{Regresión y ajuste por covariables}
\end{itemize}

\hypertarget{tema-3.-estrategias-de-aleatorizaciuxf3n-y-anuxe1lisis-basado-en-diseuxf1o-2-clases-1}{%
\subsubsection{Tema 3. Estrategias de aleatorización y análisis basado
en diseño (2
clases)}\label{tema-3.-estrategias-de-aleatorizaciuxf3n-y-anuxe1lisis-basado-en-diseuxf1o-2-clases-1}}

\begin{itemize}
\tightlist
\item
  \emph{Poder estadístico. Fórmula analítica y simulaciones.}
\item
  \emph{Uso de covariables en el diseño. Asignación por bloques y
  análisis}
\item
  \emph{Asignación aleatoria por clusters y análisis}
\end{itemize}

\hypertarget{tema-4.-diseuxf1os-experimentales-4-clases-1}{%
\subsubsection{Tema 4. Diseños experimentales (4
clases)}\label{tema-4.-diseuxf1os-experimentales-4-clases-1}}

\begin{itemize}
\tightlist
\item
  \emph{Diseños de aliento y cumplimiento imperfecto}
\item
  \emph{Diseños multi-rama}
\item
  \emph{Diseños factoriales}
\item
  \emph{Diseños de mediación}
\item
  \emph{Diseños para estimar efectos de derrame}
\item
  \emph{Diseños adaptativos}
\item
  \emph{Diagnóstico de diseños}
\end{itemize}

\hypertarget{tema-5.-anuxe1lisis-estaduxedstico-de-experimentos-3-clases-1}{%
\subsubsection{Tema 5. Análisis estadístico de experimentos (3
clases)}\label{tema-5.-anuxe1lisis-estaduxedstico-de-experimentos-3-clases-1}}

\begin{itemize}
\tightlist
\item
  \emph{Análisis de efectos heterogéneos y variables pre-tratamiento}
\item
  \emph{Corrección por hipótesis múltiples}
\item
  \emph{Spillovers (contaminación)}
\item
  \emph{Non-compliance}
\item
  \emph{Atrición}
\end{itemize}

\hypertarget{tema-6.-tipos-de-experimentos-en-ciencias-sociales-3-clases-1}{%
\subsubsection{Tema 6. Tipos de experimentos en ciencias sociales (3
clases)}\label{tema-6.-tipos-de-experimentos-en-ciencias-sociales-3-clases-1}}

\begin{itemize}
\tightlist
\item
  \emph{Experimentos de campo}
\item
  \emph{Experimentos naturales}
\item
  \emph{Experimentos de encuesta}
\item
  \emph{Experimentos de laboratorio}
\end{itemize}

\hypertarget{tema-7.-pre-registro-de-experimentos-1-clase-1}{%
\subsubsection{Tema 7. Pre-registro de experimentos (1
clase)}\label{tema-7.-pre-registro-de-experimentos-1-clase-1}}

\begin{itemize}
\tightlist
\item
  \emph{Elaboración de un plan de análisis}
\item
  \emph{Ética en la investigación experimental}
\item
  \emph{Materiales de pre-registro}
\end{itemize}

\hypertarget{lecturas-por-tema-1}{%
\subsection{Lecturas por tema}\label{lecturas-por-tema-1}}

\hypertarget{tema-1.-inferencia-causal-1}{%
\subsubsection{Tema 1. Inferencia
causal}\label{tema-1.-inferencia-causal-1}}

\begin{itemize}
\tightlist
\item
  \emph{El modelo de causalidad Neyman--Rubin}
\item
  \emph{Experimentos aleatorizados y validez}
\item
  \emph{Asignación aleatoria simple}
\item
  \emph{Estimandos (ATE, ITT, CACE, SATE, PATE, ATT, CATE, mediación)}
\end{itemize}

FEDAI, capítulos 1-2

CISSBS, capítulos 1-2.

\href{https://egap.org/resource/10-types-of-treatment-effect-you-should-know-about/}{10
Types of Treatment Effect You Should Know About}

Aronow, P. M. and Samii, C. (2016). ``Does regression produce
representative estimates of causal effects?'' \emph{American Journal of
Political Science}, 60(1):250--267

Barabas, J. and Jerit, J. (2010). ``Are survey experiments externally
valid?'' \emph{American Political Science Review}, 104(2):226--242

\hypertarget{tema-2.-anuxe1lisis-bajo-asignaciuxf3n-aleatoria-simple-1}{%
\subsubsection{Tema 2. Análisis bajo asignación aleatoria
simple}\label{tema-2.-anuxe1lisis-bajo-asignaciuxf3n-aleatoria-simple-1}}

\begin{itemize}
\tightlist
\item
  \emph{Inferencia de aleatorización (p-valores exactos para hipótesis
  nulas nítidas)}
\end{itemize}

FEDAI, capítulo 3

CISSBS capítulos 5 y 6, capítulo 9 (secciones 9.3 y 9.8), capítulo 10
(sección 10.3)

Duflo, E., Glennerster, R., \& Kremer, M. (2007). Using randomization in
development economics research: A toolkit. Handbook of development
economics, 4, 3895-3962. (capítulos 4 y 7)

EGAP,
\href{https://egap.org/resource/10-things-to-know-about-hypothesis-testing/}{10
Things to Know About Randomization Inference}

Peng Ding, Avi Feller, and Luke Miratrix (2016), ``Randomization
Inference for Treatment Effect Variation,'' Journal of the Royal
Statistical Society, Series B 78: 655--671

\begin{itemize}
\tightlist
\item
  \emph{Regresión y ajuste por covariables}
\end{itemize}

FEDAI, secciones 4.1, 4.2, y 4.3.

Freedman, David A. 2008. ``On Regression Adjustments in Experiments with
Several Treatments.'' \emph{Annals of Applied Statistics} 2(1): 176--96.

Lin, Winston. 2013. ``Agnostic Notes on Regression Adjustments to
Experimental Data: Reexamining Freedman's Critique.'' \emph{The Annals
of Applied Statistics} 7(1): 295--318.

Wager, Stefan, Wenfei Du, Jonathan Taylor, and Robert J. Tibshirani.
2016. ``High-Dimensional Regression Adjustments in Randomized
Experiments.'' \emph{Proceedings of the National Academy of Sciences of
the United States of America} 113(45): 12673--78.

\hypertarget{tema-3.-estrategias-de-aleatorizaciuxf3n-y-anuxe1lisis-basado-en-diseuxf1o-1}{%
\subsubsection{Tema 3. Estrategias de aleatorización y análisis basado
en
diseño}\label{tema-3.-estrategias-de-aleatorizaciuxf3n-y-anuxe1lisis-basado-en-diseuxf1o-1}}

\begin{itemize}
\tightlist
\item
  \emph{Poder estadístico. Fórmula analítica y simulaciones.}
\end{itemize}

EGAP,
\href{https://egap.org/resource/10-things-to-know-about-statistical-power/}{10
Things to Know About Statistical Power}

\begin{itemize}
\tightlist
\item
  \emph{Uso de covariables en el diseño. Asignación por bloques y
  análisis}
\end{itemize}

FEDAI, capítulos 3.6.1, 4.4, y 4.5.

EGAP,
\href{https://egap.org/resource/10-things-to-know-about-multisite-or-block-randomized-trials/}{10
Things to Know About Multisite or Block-Randomized Trials}

Moore, Ryan T. 2012. ``Multivariate Continuous Blocking to Improve
Political Science Experiments.'' \emph{Political Analysis} 20(4):
460--79.

Moore, Ryan T., and Sally A. Moore. 2013. ``Blocking for Sequential
Political Experiments.'' \emph{Political Analysis} 21(4): 507--23.

Coppock, Alexander. 2019. ``Randomizr: Easy-to-Use Tools for Common
Forms of Random Assignment and Sampling.''

Moore, Ryan T., and Keith Schnakenberg. 2016. ``BlockTools: Blocking,
Assignment, and Diagnosing Interference in Randomized Experiments.''

\begin{itemize}
\tightlist
\item
  \emph{Asignación aleatoria por clusters y análisis}
\end{itemize}

FEDAI, capítulo 3.6.2.

EGAP,
\href{https://egap.org/resource/10-things-to-know-about-cluster-randomization/}{10
Things to Know About Cluster Randomization}

\hypertarget{tema-4.-diseuxf1os-experimentales-1}{%
\subsubsection{Tema 4. Diseños
experimentales}\label{tema-4.-diseuxf1os-experimentales-1}}

\begin{itemize}
\tightlist
\item
  \emph{Diseños de aliento y cumplimiento imperfecto}
\end{itemize}

FEDAI, capítulos 5-6.

Athey, Susan, and Guido W Imbens. 2017. ``The Econometrics of Randomized
Experiments.'' In Handbook of Economic Field Experiments, Elsevier,
73--140.

\begin{itemize}
\tightlist
\item
  \emph{Diseños multi-rama}
\end{itemize}

Gerber, Alan S, and Donald P Green. 2012. \emph{Field Experiments:
Design, Analysis, and Interpretation}. WW Norton.

Athey, Susan, and Guido W Imbens. 2017. ``The Econometrics of Randomized
Experiments.'' In Handbook of Economic Field Experiments, Elsevier,
73--140.

\begin{itemize}
\tightlist
\item
  \emph{Diseños factoriales}
\end{itemize}

Egami, Naoki, and Kosuke Imai. 2018. ``Causal Interaction in Factorial
Experiments: Application to Conjoint Analysis.'' \emph{Journal of the
American Statistical Association}: 1--34.

Tom Pepinsky,
\href{https://tompepinsky.com/2022/10/19/factorial-experiments-conjoints-ames-and-amces/}{Factorial
Experiments, Conjoints, AMEs, and AMCEs}

\begin{itemize}
\tightlist
\item
  \emph{Diseños de mediación}
\end{itemize}

FEDAI, capítulo 10.

Imai, Kosuke, Luke Keele, Dustin Tingley, and Teppei Yamamoto. 2011.
``Unpacking the Black Box of Causality: Learning about Causal Mechanisms
from Experimental and Observational Studies.'' \emph{American Political
Science Review} 105(4): 765--89.

Imai, Kosuke, Dustin Tingley, and Teppei Yamamoto. 2013. ``Experimental
Designs for Identifying Causal Mechanisms.'' \emph{Journal of the Royal
Statistical Society. Series A: Statistics in Society} 176(1): 5--51.

Imai, Kosuke, and Teppei Yamamoto. 2013. ``Identification and
Sensitivity Analysis for Multiple Causal Mechanisms: Revisiting Evidence
from Framing Experiments.'' \emph{Political Analysis} 21(02): 141--71.

Tingley, Dustin et al.~2014. ``Mediation: R Package for Causal Mediation
Analysis.'' Journal of Statistical Software 59(5): 1--38.

\begin{itemize}
\tightlist
\item
  \emph{Diseños para estimar efectos de derrame}
\end{itemize}

Nickerson, D. W. (2008). ``Is voting contagious? Evidence from two field
experiments.'' \emph{American political Science Review}, 49-57.

Sinclair, B., McConnell, M., \& Green, D. P. (2012). ``Detecting
spillover effects: Design and analysis of multilevel experiments.''
\emph{American Journal of Political Science}, 56(4), 1055-1069.

\begin{itemize}
\tightlist
\item
  \emph{Diseños adaptativos}
\end{itemize}

Offer-westort, Molly, and Alexander Coppock. 2018. ``Adaptive
Experimental Design : Prospects and Applications in Political Science
Prepared for Presentation at the Annual Meeting of the American
Political.''

\begin{itemize}
\tightlist
\item
  \emph{Diagnóstico de diseños y potencia estadística}
\end{itemize}

Blair, Graeme, Jasper Cooper, Alexander Coppock, and Macartan Humphreys.
2019. ``Declaring and Diagnosing Research Designs.'' \emph{American
Political Science Review} 113(3): 838--59.

\hypertarget{tema-5.-anuxe1lisis-estaduxedstico-de-experimentos-1}{%
\subsubsection{Tema 5. Análisis estadístico de
experimentos}\label{tema-5.-anuxe1lisis-estaduxedstico-de-experimentos-1}}

\begin{itemize}
\tightlist
\item
  \emph{Análisis de efectos heterogéneos y variables pre-tratamiento}
\end{itemize}

FEDAI, capítulo 9.

EGAP,
\href{https://egap.org/resource/10-things-to-know-about-heterogeneous-treatment-effects/}{10
Things to Know About Heterogeneous Treatment Effects}

Sesgo post-tratamiento:

Montgomery, Jacob M., Brendan Nyhan, and Michelle Torres. 2018. ``How
Conditioning on Posttreatment Variables Can Ruin Your Experiment and
What to Do about It.'' \emph{American Journal of Political Science}
62(3): 760--75.

Variables pre-tratamiento:

Carneiro, Pedro, Sokbae Lee, and Daniel Wilhelm. 2020. ``Optimal Data
Collection for Randomized Control Trials.'' \emph{Econometrics Journal}
23(1): 1--31.

Heterogeneidad con aprendizaje automático y alta dimensionalidad:

Egami, Naoki, and Kosuke Imai. 2018. ``Causal Interaction in Factorial
Experiments: Application to Conjoint Analysis.'' \emph{Journal of the
American Statistical Association}: 1--34.

Athey, Susan, and Guido W. Imbens. 2019. ``Machine Learning Methods That
Economists Should Know About.'' \emph{Annual Review of Economics} 11(1):
685--725.

Athey, Susan, Julie Tibshirani, and Stefan Wager. 2019. ``Generalized
Random Forests.'' \emph{Annals of Statistics} 47(2): 1179--1203.

Wager, Stefan, and Susan Athey. 2018. ``Estimation and Inference of
Heterogeneous Treatment Effects Using Random Forests.'' \emph{Journal of
the American Statistical Association} 113(523): 1228--42.

Athey, Susan, and Stefan Wager. 2019. ``Estimating Treatment Effects
with Causal Forests: An Application.'' \emph{Observational Studies}
5(2): 37--51.

Xinkun Nie and Stefan Wager. 2021. Quasi-Oracle Estimation of
Heterogeneous Treatment Effects. \emph{Biometrika}, 108(2).

\begin{itemize}
\tightlist
\item
  \emph{Corrección por hipótesis múltiples}
\end{itemize}

EGAP,
\href{https://egap.org/resource/10-things-to-know-about-multiple-comparisons/}{10
Things to Know About Multiple Comparisons}

\hypertarget{tema-6.-tipos-de-experimentos-en-ciencias-sociales-1}{%
\subsubsection{Tema 6. Tipos de experimentos en ciencias
sociales}\label{tema-6.-tipos-de-experimentos-en-ciencias-sociales-1}}

\begin{itemize}
\tightlist
\item
  \emph{Experimentos naturales}
\end{itemize}

Dunning, Thad. 2012. \emph{Natural Experiments in the Social Sciences
Natural Experiments in the Social Sciences}.

Dunning, Thad. 2008. ``Improving Causal Inference: Strengths and
Limitations of Natural Experiments.'' \emph{Political Research
Quarterly} 61(2): 282--293.

\begin{itemize}
\tightlist
\item
  \emph{Experimentos de campo}
\end{itemize}

Gerber, Alan S, and Donald P Green. 2012. \emph{Field Experiments:
Design, Analysis, and Interpretation}. WW Norton.

Coppock, A. and Green, D. P. (2015). ``Assessing the correspondence
between experimental results obtained in the lab and field: A review of
recent social science research''. \emph{Political Science Research and
Methods}, 3(1):113--131

List, J. A. (2011). ``Why economists should conduct field experiments
and 14 tips for pulling one off''. \emph{The Journal of Economic
Perspectives}, 25(3):3--15

\begin{itemize}
\tightlist
\item
  \emph{Experimentos de encuesta}
\end{itemize}

Generales:

Gaines, Brian J., James H. Kuklinski, and Paul J. Quirk. 2007. ``The
Logic of the Survey Experiment Reexamined.'' \emph{Political Analysis}
15(01): 1--20.

Coppock, Alexander. 2018. ``Generalizing from Survey Experiments
Conducted on Mechanical Turk: A Replication Approach.'' \emph{Political
Science Research and Methods} (2015): 1--16.

Tobergte, David R., and Shirley Curtis. 2013. ``The Generalizability of
Survey Experiments.'' \emph{Journal of Chemical Information and
Modeling} 53(9): 1689--99.

Tratamientos de información

Linos, Katerina, and Kimberly Twist. 2018. ``Diverse Pre-Treatment
Effects in Survey Experiments.'' \emph{Journal of Experimental Political
Science} 5(2): 148--58.

Fernández-Albertos, José, and Alexander Kuo. 2018. ``Income Perception,
Information, and Progressive Taxation: Evidence from a Survey
Experiment.'' \emph{Political Science Research and Methods} 6(01):
83--110.

Cruces, Guillermo, Ricardo Perez-Truglia, and Martin Tetaz. 2013.
``Biased Perceptions of Income Distribution and Preferences for
Redistribution: Evidence from a Survey Experiment.'' \emph{Journal of
Public Economics} 98: 100--112.

Stokes, Leah C., and Christopher Warshaw. 2017. ``Renewable Energy
Policy Design and Framing Influence Public Support in the United
States.'' Nature Energy 2: 17107.

Conjoint:

Hainmueller, Jens, Daniel J. Hopkins, and Teppei Yamamoto. 2014.
``Causal Inference in Conjoint Analysis: Understanding Multidimensional
Choices via Stated Preference Experiments.'' \emph{Political Analysis}
22(1): 1--30.

Leeper, Thomas J., Sara B. Hobolt, and James Tilley. 2020. ``Measuring
Subgroup Preferences in Conjoint Experiments.'' \emph{Political
Analysis} 28(2): 207--21.

Experimentos de lista:

Blair, Graeme, and Kosuke Imai. 2012. ``Statistical Analysis of List
Experiments.'' \emph{Political Analysis} 20(1): 47--77.

Mecanismos causales:

Acharya, Avidit, Matthew Blackwell, and Maya Sen.~2018. ``Analyzing
Causal Mechanisms in Survey Experiments.'' \emph{Political Analysis}
26(4): 1--31.

Cumplimiento a la asignación y atención:

Aronow, Peter M., Jonathon Baron, and Lauren Pinson. 2019. ``A Note on
Dropping Experimental Subjects Who Fail a Manipulation Check.''
\emph{Political Analysis} 27(4): 572--89.

\begin{itemize}
\tightlist
\item
  \emph{Experimentos de laboratorio}
\end{itemize}

Levitt, Steven, D., and John A. List. 2007. ``What Do Laboratory
Experiments Measuring Social Preferences Reveal About the Real World?''
Journal of Economic Perspectives, 21 (2): 153-174.

Falk, A. and Heckman, J. J. (2009). ``Lab experiments are a major source
of knowledge in the social sciences''. \emph{Science},
326(5952):535--538

Habyarimana, J., Humphreys, M., Posner, D. N., and Weinstein, J. M.
(2007). ``Why does ethnic diversity undermine public goods provision?''
\emph{American Political Science Review}, 101(4):709--725

Oxley, D. R., Smith, K. B., Alford, J. R., Hibbing, M. V., Miller, J.
L., Scalora, M., Hatemi, P. K., and Hibbing, J. R. (2008). ``Political
attitudes vary with physiological traits''. \emph{Science},
321(5896):1667--1670

Levine, D. K. and Palfrey, T. R. (2007). The paradox of voter
participation? a laboratory study. American political science Review,
101(1):143--158

\hypertarget{tema-7.-pre-registro-de-experimentos-1}{%
\subsubsection{Tema 7. Pre-registro de
experimentos}\label{tema-7.-pre-registro-de-experimentos-1}}

\begin{itemize}
\tightlist
\item
  \emph{Elaboración de un plan de análisis}
\item
  \emph{Ética en la investigación experimental}
\item
  \emph{Materiales de pre-registro}
\end{itemize}

Bowers, J., \& Testa, P. F. (2019). ``Better Government, Better Science:
The Promise of and Challenges Facing the Evidence-Informed Policy
Movement.'' \emph{Annual Review of Political Science}, 22, 521-542.

\begin{quote}
\begin{quote}
\begin{quote}
\begin{quote}
\begin{quote}
\begin{quote}
\begin{quote}
d29c5cead8517ee1de2b99fe258309fbfe5e2fb3
\end{quote}
\end{quote}
\end{quote}
\end{quote}
\end{quote}
\end{quote}
\end{quote}

\end{document}
