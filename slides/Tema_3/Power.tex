% Options for packages loaded elsewhere
\PassOptionsToPackage{unicode}{hyperref}
\PassOptionsToPackage{hyphens}{url}
%
\documentclass[
  ignorenonframetext,
]{beamer}
\usepackage{pgfpages}
\setbeamertemplate{caption}[numbered]
\setbeamertemplate{caption label separator}{: }
\setbeamercolor{caption name}{fg=normal text.fg}
\beamertemplatenavigationsymbolsempty
% Prevent slide breaks in the middle of a paragraph
\widowpenalties 1 10000
\raggedbottom
\setbeamertemplate{part page}{
  \centering
  \begin{beamercolorbox}[sep=16pt,center]{part title}
    \usebeamerfont{part title}\insertpart\par
  \end{beamercolorbox}
}
\setbeamertemplate{section page}{
  \centering
  \begin{beamercolorbox}[sep=12pt,center]{part title}
    \usebeamerfont{section title}\insertsection\par
  \end{beamercolorbox}
}
\setbeamertemplate{subsection page}{
  \centering
  \begin{beamercolorbox}[sep=8pt,center]{part title}
    \usebeamerfont{subsection title}\insertsubsection\par
  \end{beamercolorbox}
}
\AtBeginPart{
  \frame{\partpage}
}
\AtBeginSection{
  \ifbibliography
  \else
    \frame{\sectionpage}
  \fi
}
\AtBeginSubsection{
  \frame{\subsectionpage}
}
\usepackage{lmodern}
\usepackage{amssymb,amsmath}
\usepackage{ifxetex,ifluatex}
\ifnum 0\ifxetex 1\fi\ifluatex 1\fi=0 % if pdftex
  \usepackage[T1]{fontenc}
  \usepackage[utf8]{inputenc}
  \usepackage{textcomp} % provide euro and other symbols
\else % if luatex or xetex
  \usepackage{unicode-math}
  \defaultfontfeatures{Scale=MatchLowercase}
  \defaultfontfeatures[\rmfamily]{Ligatures=TeX,Scale=1}
\fi
\usetheme[]{default}
\usefonttheme{professionalfonts}
% Use upquote if available, for straight quotes in verbatim environments
\IfFileExists{upquote.sty}{\usepackage{upquote}}{}
\IfFileExists{microtype.sty}{% use microtype if available
  \usepackage[]{microtype}
  \UseMicrotypeSet[protrusion]{basicmath} % disable protrusion for tt fonts
}{}
\makeatletter
\@ifundefined{KOMAClassName}{% if non-KOMA class
  \IfFileExists{parskip.sty}{%
    \usepackage{parskip}
  }{% else
    \setlength{\parindent}{0pt}
    \setlength{\parskip}{6pt plus 2pt minus 1pt}}
}{% if KOMA class
  \KOMAoptions{parskip=half}}
\makeatother
\usepackage{xcolor}
\IfFileExists{xurl.sty}{\usepackage{xurl}}{} % add URL line breaks if available
\IfFileExists{bookmark.sty}{\usepackage{bookmark}}{\usepackage{hyperref}}
\hypersetup{
  pdftitle={Potencia estadística},
  hidelinks,
  pdfcreator={LaTeX via pandoc}}
\urlstyle{same} % disable monospaced font for URLs
\newif\ifbibliography
\usepackage{color}
\usepackage{fancyvrb}
\newcommand{\VerbBar}{|}
\newcommand{\VERB}{\Verb[commandchars=\\\{\}]}
\DefineVerbatimEnvironment{Highlighting}{Verbatim}{commandchars=\\\{\}}
% Add ',fontsize=\small' for more characters per line
\usepackage{framed}
\definecolor{shadecolor}{RGB}{248,248,248}
\newenvironment{Shaded}{\begin{snugshade}}{\end{snugshade}}
\newcommand{\AlertTok}[1]{\textcolor[rgb]{0.94,0.16,0.16}{#1}}
\newcommand{\AnnotationTok}[1]{\textcolor[rgb]{0.56,0.35,0.01}{\textbf{\textit{#1}}}}
\newcommand{\AttributeTok}[1]{\textcolor[rgb]{0.77,0.63,0.00}{#1}}
\newcommand{\BaseNTok}[1]{\textcolor[rgb]{0.00,0.00,0.81}{#1}}
\newcommand{\BuiltInTok}[1]{#1}
\newcommand{\CharTok}[1]{\textcolor[rgb]{0.31,0.60,0.02}{#1}}
\newcommand{\CommentTok}[1]{\textcolor[rgb]{0.56,0.35,0.01}{\textit{#1}}}
\newcommand{\CommentVarTok}[1]{\textcolor[rgb]{0.56,0.35,0.01}{\textbf{\textit{#1}}}}
\newcommand{\ConstantTok}[1]{\textcolor[rgb]{0.00,0.00,0.00}{#1}}
\newcommand{\ControlFlowTok}[1]{\textcolor[rgb]{0.13,0.29,0.53}{\textbf{#1}}}
\newcommand{\DataTypeTok}[1]{\textcolor[rgb]{0.13,0.29,0.53}{#1}}
\newcommand{\DecValTok}[1]{\textcolor[rgb]{0.00,0.00,0.81}{#1}}
\newcommand{\DocumentationTok}[1]{\textcolor[rgb]{0.56,0.35,0.01}{\textbf{\textit{#1}}}}
\newcommand{\ErrorTok}[1]{\textcolor[rgb]{0.64,0.00,0.00}{\textbf{#1}}}
\newcommand{\ExtensionTok}[1]{#1}
\newcommand{\FloatTok}[1]{\textcolor[rgb]{0.00,0.00,0.81}{#1}}
\newcommand{\FunctionTok}[1]{\textcolor[rgb]{0.00,0.00,0.00}{#1}}
\newcommand{\ImportTok}[1]{#1}
\newcommand{\InformationTok}[1]{\textcolor[rgb]{0.56,0.35,0.01}{\textbf{\textit{#1}}}}
\newcommand{\KeywordTok}[1]{\textcolor[rgb]{0.13,0.29,0.53}{\textbf{#1}}}
\newcommand{\NormalTok}[1]{#1}
\newcommand{\OperatorTok}[1]{\textcolor[rgb]{0.81,0.36,0.00}{\textbf{#1}}}
\newcommand{\OtherTok}[1]{\textcolor[rgb]{0.56,0.35,0.01}{#1}}
\newcommand{\PreprocessorTok}[1]{\textcolor[rgb]{0.56,0.35,0.01}{\textit{#1}}}
\newcommand{\RegionMarkerTok}[1]{#1}
\newcommand{\SpecialCharTok}[1]{\textcolor[rgb]{0.00,0.00,0.00}{#1}}
\newcommand{\SpecialStringTok}[1]{\textcolor[rgb]{0.31,0.60,0.02}{#1}}
\newcommand{\StringTok}[1]{\textcolor[rgb]{0.31,0.60,0.02}{#1}}
\newcommand{\VariableTok}[1]{\textcolor[rgb]{0.00,0.00,0.00}{#1}}
\newcommand{\VerbatimStringTok}[1]{\textcolor[rgb]{0.31,0.60,0.02}{#1}}
\newcommand{\WarningTok}[1]{\textcolor[rgb]{0.56,0.35,0.01}{\textbf{\textit{#1}}}}
\setlength{\emergencystretch}{3em} % prevent overfull lines
\providecommand{\tightlist}{%
  \setlength{\itemsep}{0pt}\setlength{\parskip}{0pt}}
\setcounter{secnumdepth}{-\maxdimen} % remove section numbering
\usepackage{fancyhdr}
\usepackage{lastpage}
\setbeamertemplate{navigation symbols}{}
\setbeamertemplate{footline}[page number]
\pagenumbering{arabic}
% \usepackage[mathbf,mathcal]{euler}
\usepackage{multicol}


\newenvironment{cols}[1][]{}{}

\newenvironment{col}[1]{\begin{minipage}{#1}\ignorespaces}{%
\end{minipage}
\ifhmode\unskip\fi
\aftergroup\useignorespacesandallpars}

\def\useignorespacesandallpars#1\ignorespaces\fi{%
#1\fi\ignorespacesandallpars}

\makeatletter
\def\ignorespacesandallpars{%
  \@ifnextchar\par
    {\expandafter\ignorespacesandallpars\@gobble}%
    {}%
}
\makeatother

\title{Potencia estadística}
\author{Diseño e implementación de experimentos en ciencias sociales\\
\emph{Departamento de Economía (UdelaR)}}
\date{}

\begin{document}
\frame{\titlepage}

\hypertarget{what-is-power}{%
\section{What is power?}\label{what-is-power}}

\begin{frame}{What is power?}
\protect\hypertarget{what-is-power-1}{}
\begin{itemize}
\item
  We want to separate signal from noise.
\item
  Power = probability of rejecting null hypothesis, given true effect
  \(\ne\) 0.
\item
  In other words, it is the ability to detect an effect given that it
  exists.
\item
  Formally: (1 - Type II) error rate.
\item
  Thus, power \(\in\) (0, 1).
\item
  Standard thresholds: 0.8 or 0.9.
\end{itemize}
\end{frame}

\begin{frame}{Starting point for power analysis}
\protect\hypertarget{starting-point-for-power-analysis}{}
\begin{itemize}
\item
  Power analysis is something we do \emph{before} we run a study.

  \begin{itemize}
  \item
    Helps you figure out the sample you need to detect a given effect
    size.
  \item
    Or helps you figure out a minimal detectable difference given a set
    sample size.
  \item
    May help you decide whether to run a study.
  \end{itemize}
\item
  It is hard to learn from an under-powered null finding.

  \begin{itemize}
  \tightlist
  \item
    Was there an effect, but we were unable to detect it? or was there
    no effect? We can't say.
  \end{itemize}
\end{itemize}
\end{frame}

\begin{frame}{Power}
\protect\hypertarget{power}{}
\begin{itemize}
\item
  Say there truly is a treatment effect and you run your experiment many
  times. How often will you get a statistically significant result?
\item
  Some guesswork to answer this question.

  \begin{itemize}
  \item
    How big is your treatment effect?
  \item
    How many units are treated, measured?
  \item
    How much noise is there in the measurement of your outcome?
  \end{itemize}
\end{itemize}
\end{frame}

\begin{frame}{Approaches to power calculation}
\protect\hypertarget{approaches-to-power-calculation}{}
\begin{itemize}
\item
  Analytical calculations of power
\item
  Simulation
\end{itemize}
\end{frame}

\begin{frame}{Power calculation tools}
\protect\hypertarget{power-calculation-tools}{}
\begin{itemize}
\item
  Interactive

  \begin{itemize}
  \item
    \href{https://egap.shinyapps.io/power-app/}{EGAP Power Calculator}
  \item
    \href{https://rpsychologist.com/d3/NHST/}{rpsychologist}
  \end{itemize}
\item
  R Packages

  \begin{itemize}
  \item
    \href{https://cran.r-project.org/web/packages/pwr/index.html}{pwr}
  \item
    \href{https://cran.r-project.org/web/packages/DeclareDesign/index.html}{DeclareDesign},
    see also \url{https://declaredesign.org/}
  \end{itemize}
\end{itemize}
\end{frame}

\hypertarget{analytical-calculations-of-power}{%
\section{Analytical calculations of
power}\label{analytical-calculations-of-power}}

\begin{frame}{Analytical calculations of power}
\protect\hypertarget{analytical-calculations-of-power-1}{}
\begin{itemize}
\item
  Formula: \begin{align*}
  \text{Power} &= \Phi\left(\frac{|\tau| \sqrt{N}}{2\sigma}- \Phi^{-1}(1- \frac{\alpha}{2})\right)
  \end{align*}
\item
  Components:

  \begin{itemize}
  \tightlist
  \item
    \(\phi\): standard normal CDF is monotonically increasing
  \item
    \(\tau\): the effect size
  \item
    \(N\): the sample size
  \item
    \(\sigma\): the standard deviation of the outcome
  \item
    \(\alpha\): the significance level (typically 0.05)
  \end{itemize}
\end{itemize}
\end{frame}

\begin{frame}[fragile]{Example: Simulation-based power for complete
randomization}
\protect\hypertarget{example-simulation-based-power-for-complete-randomization}{}
\begin{Shaded}
\begin{Highlighting}[]
\NormalTok{power\_calculator \textless{}{-}}\StringTok{ }\ControlFlowTok{function}\NormalTok{(mu\_t, mu\_c,}
\NormalTok{    sigma, }\DataTypeTok{alpha =} \FloatTok{0.05}\NormalTok{, N) \{}
\NormalTok{    lowertail \textless{}{-}}\StringTok{ }\NormalTok{(}\KeywordTok{abs}\NormalTok{(mu\_t }\OperatorTok{{-}}\StringTok{ }\NormalTok{mu\_c) }\OperatorTok{*}\StringTok{ }\KeywordTok{sqrt}\NormalTok{(N))}\OperatorTok{/}\NormalTok{(}\DecValTok{2} \OperatorTok{*}
\StringTok{        }\NormalTok{sigma)}
\NormalTok{    uppertail \textless{}{-}}\StringTok{ }\DecValTok{{-}1} \OperatorTok{*}\StringTok{ }\NormalTok{lowertail}
\NormalTok{    beta \textless{}{-}}\StringTok{ }\KeywordTok{pnorm}\NormalTok{(lowertail }\OperatorTok{{-}}\StringTok{ }\KeywordTok{qnorm}\NormalTok{(}\DecValTok{1} \OperatorTok{{-}}\StringTok{ }\NormalTok{alpha}\OperatorTok{/}\DecValTok{2}\NormalTok{),}
        \DataTypeTok{lower.tail =} \OtherTok{TRUE}\NormalTok{)}
    \OperatorTok{+}\DecValTok{1} \OperatorTok{{-}}\StringTok{ }\KeywordTok{pnorm}\NormalTok{(uppertail }\OperatorTok{{-}}\StringTok{ }\KeywordTok{qnorm}\NormalTok{(}\DecValTok{1} \OperatorTok{{-}}\StringTok{ }\NormalTok{alpha}\OperatorTok{/}\DecValTok{2}\NormalTok{),}
        \DataTypeTok{lower.tail =} \OtherTok{FALSE}\NormalTok{)}
    \KeywordTok{return}\NormalTok{(beta)}
\NormalTok{\}}
\end{Highlighting}
\end{Shaded}
\end{frame}

\begin{frame}[fragile]{Example: Simulation-based power for complete
randomization}
\protect\hypertarget{example-simulation-based-power-for-complete-randomization-1}{}
\begin{Shaded}
\begin{Highlighting}[]
\KeywordTok{power\_calculator}\NormalTok{(}\DataTypeTok{mu\_t =} \DecValTok{1}\OperatorTok{/}\DecValTok{4}\NormalTok{, }\DataTypeTok{mu\_c =} \DecValTok{0}\NormalTok{, }\DataTypeTok{sigma =} \DecValTok{1}\NormalTok{,}
    \DataTypeTok{alpha =} \FloatTok{0.05}\NormalTok{, }\DataTypeTok{N =} \DecValTok{100}\NormalTok{)}
\end{Highlighting}
\end{Shaded}

\begin{verbatim}
## [1] 0.2388632
\end{verbatim}
\end{frame}

\begin{frame}[fragile]{Example: Simulation-based power for complete
randomization}
\protect\hypertarget{example-simulation-based-power-for-complete-randomization-2}{}
\begin{Shaded}
\begin{Highlighting}[]
\KeywordTok{power\_calculator}\NormalTok{(}\DataTypeTok{mu\_t =} \DecValTok{1}\OperatorTok{/}\DecValTok{4}\NormalTok{, }\DataTypeTok{mu\_c =} \DecValTok{0}\NormalTok{, }\DataTypeTok{sigma =} \DecValTok{1}\NormalTok{,}
    \DataTypeTok{alpha =} \FloatTok{0.05}\NormalTok{, }\DataTypeTok{N =} \DecValTok{1000}\NormalTok{)}
\end{Highlighting}
\end{Shaded}

\begin{verbatim}
## [1] 0.9768629
\end{verbatim}
\end{frame}

\begin{frame}[fragile]{Example: Simulation-based power for complete
randomization}
\protect\hypertarget{example-simulation-based-power-for-complete-randomization-3}{}
\begin{Shaded}
\begin{Highlighting}[]
\KeywordTok{power\_calculator}\NormalTok{(}\DataTypeTok{mu\_t =} \DecValTok{1}\OperatorTok{/}\DecValTok{4}\NormalTok{, }\DataTypeTok{mu\_c =} \DecValTok{0}\NormalTok{, }\DataTypeTok{sigma =} \DecValTok{2}\NormalTok{,}
    \DataTypeTok{alpha =} \FloatTok{0.05}\NormalTok{, }\DataTypeTok{N =} \DecValTok{1000}\NormalTok{)}
\end{Highlighting}
\end{Shaded}

\begin{verbatim}
## [1] 0.5065661
\end{verbatim}
\end{frame}

\begin{frame}[fragile]{Example: Simulation-based power for complete
randomization}
\protect\hypertarget{example-simulation-based-power-for-complete-randomization-4}{}
\begin{Shaded}
\begin{Highlighting}[]
\KeywordTok{power\_calculator}\NormalTok{(}\DataTypeTok{mu\_t =} \DecValTok{2}\OperatorTok{/}\DecValTok{4}\NormalTok{, }\DataTypeTok{mu\_c =} \DecValTok{0}\NormalTok{, }\DataTypeTok{sigma =} \DecValTok{1}\NormalTok{,}
    \DataTypeTok{alpha =} \FloatTok{0.05}\NormalTok{, }\DataTypeTok{N =} \DecValTok{1000}\NormalTok{)}
\end{Highlighting}
\end{Shaded}

\begin{verbatim}
## [1] 1
\end{verbatim}
\end{frame}

\begin{frame}[fragile]{Example: Using DeclareDesign}
\protect\hypertarget{example-using-declaredesign}{}
\begin{Shaded}
\begin{Highlighting}[]
\KeywordTok{library}\NormalTok{(DeclareDesign)}
\KeywordTok{library}\NormalTok{(tidyverse)}

\NormalTok{P0 \textless{}{-}}\StringTok{ }\KeywordTok{declare\_population}\NormalTok{(N, }\DataTypeTok{u0 =} \KeywordTok{rnorm}\NormalTok{(N))}
\CommentTok{\# declare Y(Z=1) and Y(Z=0)}
\NormalTok{O0 \textless{}{-}}\StringTok{ }\KeywordTok{declare\_potential\_outcomes}\NormalTok{(}\DataTypeTok{Y\_Z\_0 =} \DecValTok{5} \OperatorTok{+}
\StringTok{    }\NormalTok{u0, }\DataTypeTok{Y\_Z\_1 =}\NormalTok{ Y\_Z\_}\DecValTok{0} \OperatorTok{+}\StringTok{ }\NormalTok{tau)}
\CommentTok{\# design is to assign m units to}
\CommentTok{\# treatment}
\NormalTok{A0 \textless{}{-}}\StringTok{ }\KeywordTok{declare\_assignment}\NormalTok{(}\DataTypeTok{Z =} \KeywordTok{conduct\_ra}\NormalTok{(}\DataTypeTok{N =}\NormalTok{ N,}
    \DataTypeTok{m =} \KeywordTok{round}\NormalTok{(N}\OperatorTok{/}\DecValTok{2}\NormalTok{)))}
\CommentTok{\# estimand is the average difference}
\CommentTok{\# between Y(Z=1) and Y(Z=0)}
\NormalTok{estimand\_ate \textless{}{-}}\StringTok{ }\KeywordTok{declare\_inquiry}\NormalTok{(}\DataTypeTok{ATE =} \KeywordTok{mean}\NormalTok{(Y\_Z\_}\DecValTok{1} \OperatorTok{{-}}
\StringTok{    }\NormalTok{Y\_Z\_}\DecValTok{0}\NormalTok{))}
\NormalTok{R0 \textless{}{-}}\StringTok{ }\KeywordTok{declare\_reveal}\NormalTok{(Y, Z)}
\NormalTok{design0\_base \textless{}{-}}\StringTok{ }\NormalTok{P0 }\OperatorTok{+}\StringTok{ }\NormalTok{A0 }\OperatorTok{+}\StringTok{ }\NormalTok{O0 }\OperatorTok{+}\StringTok{ }\NormalTok{R0}
\end{Highlighting}
\end{Shaded}
\end{frame}

\begin{frame}[fragile]{Example: Using DeclareDesign}
\protect\hypertarget{example-using-declaredesign-1}{}
\begin{Shaded}
\begin{Highlighting}[]
\CommentTok{\#\# For example:}
\NormalTok{design0\_N100\_tau25 \textless{}{-}}\StringTok{ }\KeywordTok{redesign}\NormalTok{(design0\_base,}
    \DataTypeTok{N =} \DecValTok{100}\NormalTok{, }\DataTypeTok{tau =} \FloatTok{0.25}\NormalTok{)}
\NormalTok{dat0\_N100\_tau25 \textless{}{-}}\StringTok{ }\KeywordTok{draw\_data}\NormalTok{(design0\_N100\_tau25)}
\KeywordTok{head}\NormalTok{(dat0\_N100\_tau25)}
\end{Highlighting}
\end{Shaded}

\begin{verbatim}
##    ID          u0 Z    Y_Z_0    Y_Z_1        Y
## 1 001  0.44853297 1 5.448533 5.698533 5.698533
## 2 002 -0.20105196 1 4.798948 5.048948 5.048948
## 3 003  1.54239837 0 6.542398 6.792398 6.542398
## 4 004 -0.54998139 1 4.450019 4.700019 4.700019
## 5 005 -1.10680777 1 3.893192 4.143192 4.143192
## 6 006 -0.04259078 0 4.957409 5.207409 4.957409
\end{verbatim}
\end{frame}

\begin{frame}[fragile]{Example: Using DeclareDesign}
\protect\hypertarget{example-using-declaredesign-2}{}
\begin{Shaded}
\begin{Highlighting}[]
\KeywordTok{with}\NormalTok{(dat0\_N100\_tau25, }\KeywordTok{mean}\NormalTok{(Y\_Z\_}\DecValTok{1} \OperatorTok{{-}}\StringTok{ }\NormalTok{Y\_Z\_}\DecValTok{0}\NormalTok{))  }\CommentTok{\# true ATE}
\end{Highlighting}
\end{Shaded}

\begin{verbatim}
## [1] 0.25
\end{verbatim}

\begin{Shaded}
\begin{Highlighting}[]
\KeywordTok{with}\NormalTok{(dat0\_N100\_tau25, }\KeywordTok{mean}\NormalTok{(Y[Z }\OperatorTok{==}\StringTok{ }\DecValTok{1}\NormalTok{]) }\OperatorTok{{-}}\StringTok{ }\KeywordTok{mean}\NormalTok{(Y[Z }\OperatorTok{==}
\StringTok{    }\DecValTok{0}\NormalTok{]))  }\CommentTok{\# estimate}
\end{Highlighting}
\end{Shaded}

\begin{verbatim}
## [1] 0.3374549
\end{verbatim}

\begin{Shaded}
\begin{Highlighting}[]
\KeywordTok{lm\_robust}\NormalTok{(Y }\OperatorTok{\textasciitilde{}}\StringTok{ }\NormalTok{Z, }\DataTypeTok{data =}\NormalTok{ dat0\_N100\_tau25)}\OperatorTok{$}\NormalTok{coef  }\CommentTok{\# estimate}
\end{Highlighting}
\end{Shaded}

\begin{verbatim}
## (Intercept)           Z 
##   5.0510077   0.3374549
\end{verbatim}
\end{frame}

\begin{frame}[fragile]{Example: Using DeclareDesign}
\protect\hypertarget{example-using-declaredesign-3}{}
\begin{Shaded}
\begin{Highlighting}[]
\NormalTok{E0 \textless{}{-}}\StringTok{ }\KeywordTok{declare\_estimator}\NormalTok{(Y }\OperatorTok{\textasciitilde{}}\StringTok{ }\NormalTok{Z, }\DataTypeTok{model =}\NormalTok{ lm\_robust,}
    \DataTypeTok{label =} \StringTok{"t test 1"}\NormalTok{, }\DataTypeTok{inquiry =} \StringTok{"ATE"}\NormalTok{)}
\NormalTok{t\_test \textless{}{-}}\StringTok{ }\ControlFlowTok{function}\NormalTok{(data) \{}
\NormalTok{    test \textless{}{-}}\StringTok{ }\KeywordTok{with}\NormalTok{(data, }\KeywordTok{t.test}\NormalTok{(}\DataTypeTok{x =}\NormalTok{ Y[Z }\OperatorTok{==}
\StringTok{        }\DecValTok{1}\NormalTok{], }\DataTypeTok{y =}\NormalTok{ Y[Z }\OperatorTok{==}\StringTok{ }\DecValTok{0}\NormalTok{]))}
    \KeywordTok{data.frame}\NormalTok{(}\DataTypeTok{statistic =}\NormalTok{ test}\OperatorTok{$}\NormalTok{statistic,}
        \DataTypeTok{p.value =}\NormalTok{ test}\OperatorTok{$}\NormalTok{p.value)}
\NormalTok{\}}
\NormalTok{T0 \textless{}{-}}\StringTok{ }\KeywordTok{declare\_test}\NormalTok{(}\DataTypeTok{handler =} \KeywordTok{label\_test}\NormalTok{(t\_test),}
    \DataTypeTok{label =} \StringTok{"t test 2"}\NormalTok{)}
\NormalTok{design0\_plus\_tests \textless{}{-}}\StringTok{ }\NormalTok{design0\_base }\OperatorTok{+}\StringTok{ }\NormalTok{E0 }\OperatorTok{+}
\StringTok{    }\NormalTok{T0}

\NormalTok{design0\_N100\_tau25\_plus \textless{}{-}}\StringTok{ }\KeywordTok{redesign}\NormalTok{(design0\_plus\_tests,}
    \DataTypeTok{N =} \DecValTok{100}\NormalTok{, }\DataTypeTok{tau =} \FloatTok{0.25}\NormalTok{)}

\CommentTok{\#\# Only repeat the random assignment,}
\CommentTok{\#\# not the creation of Y0. Ignore}
\CommentTok{\#\# warning}
\KeywordTok{names}\NormalTok{(design0\_N100\_tau25\_plus)}
\end{Highlighting}
\end{Shaded}

\begin{verbatim}
## [1] "P0"       "A0"       "O0"       "R0"       "t test 1" "t test 2"
\end{verbatim}

\begin{Shaded}
\begin{Highlighting}[]
\NormalTok{design0\_N100\_tau25\_sims \textless{}{-}}\StringTok{ }\KeywordTok{simulate\_design}\NormalTok{(design0\_N100\_tau25\_plus,}
    \DataTypeTok{sims =} \KeywordTok{c}\NormalTok{(}\DecValTok{1}\NormalTok{, }\DecValTok{100}\NormalTok{, }\DecValTok{1}\NormalTok{, }\DecValTok{1}\NormalTok{, }\DecValTok{1}\NormalTok{, }\DecValTok{1}\NormalTok{))  }\CommentTok{\# only repeat the random assignment}
\end{Highlighting}
\end{Shaded}

\begin{verbatim}
## Warning: We recommend you choose a number of simulations higher than 30.
\end{verbatim}

\begin{verbatim}
## Warning in fn(data, ~(Y ~ Z), model = ~lm_robust): The argument 'model = ' is
## deprecated. Please use '.method = ' instead.

## Warning in fn(data, ~(Y ~ Z), model = ~lm_robust): The argument 'model = ' is
## deprecated. Please use '.method = ' instead.

## Warning in fn(data, ~(Y ~ Z), model = ~lm_robust): The argument 'model = ' is
## deprecated. Please use '.method = ' instead.

## Warning in fn(data, ~(Y ~ Z), model = ~lm_robust): The argument 'model = ' is
## deprecated. Please use '.method = ' instead.

## Warning in fn(data, ~(Y ~ Z), model = ~lm_robust): The argument 'model = ' is
## deprecated. Please use '.method = ' instead.

## Warning in fn(data, ~(Y ~ Z), model = ~lm_robust): The argument 'model = ' is
## deprecated. Please use '.method = ' instead.

## Warning in fn(data, ~(Y ~ Z), model = ~lm_robust): The argument 'model = ' is
## deprecated. Please use '.method = ' instead.

## Warning in fn(data, ~(Y ~ Z), model = ~lm_robust): The argument 'model = ' is
## deprecated. Please use '.method = ' instead.

## Warning in fn(data, ~(Y ~ Z), model = ~lm_robust): The argument 'model = ' is
## deprecated. Please use '.method = ' instead.

## Warning in fn(data, ~(Y ~ Z), model = ~lm_robust): The argument 'model = ' is
## deprecated. Please use '.method = ' instead.

## Warning in fn(data, ~(Y ~ Z), model = ~lm_robust): The argument 'model = ' is
## deprecated. Please use '.method = ' instead.

## Warning in fn(data, ~(Y ~ Z), model = ~lm_robust): The argument 'model = ' is
## deprecated. Please use '.method = ' instead.

## Warning in fn(data, ~(Y ~ Z), model = ~lm_robust): The argument 'model = ' is
## deprecated. Please use '.method = ' instead.

## Warning in fn(data, ~(Y ~ Z), model = ~lm_robust): The argument 'model = ' is
## deprecated. Please use '.method = ' instead.

## Warning in fn(data, ~(Y ~ Z), model = ~lm_robust): The argument 'model = ' is
## deprecated. Please use '.method = ' instead.

## Warning in fn(data, ~(Y ~ Z), model = ~lm_robust): The argument 'model = ' is
## deprecated. Please use '.method = ' instead.

## Warning in fn(data, ~(Y ~ Z), model = ~lm_robust): The argument 'model = ' is
## deprecated. Please use '.method = ' instead.

## Warning in fn(data, ~(Y ~ Z), model = ~lm_robust): The argument 'model = ' is
## deprecated. Please use '.method = ' instead.

## Warning in fn(data, ~(Y ~ Z), model = ~lm_robust): The argument 'model = ' is
## deprecated. Please use '.method = ' instead.

## Warning in fn(data, ~(Y ~ Z), model = ~lm_robust): The argument 'model = ' is
## deprecated. Please use '.method = ' instead.

## Warning in fn(data, ~(Y ~ Z), model = ~lm_robust): The argument 'model = ' is
## deprecated. Please use '.method = ' instead.

## Warning in fn(data, ~(Y ~ Z), model = ~lm_robust): The argument 'model = ' is
## deprecated. Please use '.method = ' instead.

## Warning in fn(data, ~(Y ~ Z), model = ~lm_robust): The argument 'model = ' is
## deprecated. Please use '.method = ' instead.

## Warning in fn(data, ~(Y ~ Z), model = ~lm_robust): The argument 'model = ' is
## deprecated. Please use '.method = ' instead.

## Warning in fn(data, ~(Y ~ Z), model = ~lm_robust): The argument 'model = ' is
## deprecated. Please use '.method = ' instead.

## Warning in fn(data, ~(Y ~ Z), model = ~lm_robust): The argument 'model = ' is
## deprecated. Please use '.method = ' instead.

## Warning in fn(data, ~(Y ~ Z), model = ~lm_robust): The argument 'model = ' is
## deprecated. Please use '.method = ' instead.

## Warning in fn(data, ~(Y ~ Z), model = ~lm_robust): The argument 'model = ' is
## deprecated. Please use '.method = ' instead.

## Warning in fn(data, ~(Y ~ Z), model = ~lm_robust): The argument 'model = ' is
## deprecated. Please use '.method = ' instead.

## Warning in fn(data, ~(Y ~ Z), model = ~lm_robust): The argument 'model = ' is
## deprecated. Please use '.method = ' instead.

## Warning in fn(data, ~(Y ~ Z), model = ~lm_robust): The argument 'model = ' is
## deprecated. Please use '.method = ' instead.

## Warning in fn(data, ~(Y ~ Z), model = ~lm_robust): The argument 'model = ' is
## deprecated. Please use '.method = ' instead.

## Warning in fn(data, ~(Y ~ Z), model = ~lm_robust): The argument 'model = ' is
## deprecated. Please use '.method = ' instead.

## Warning in fn(data, ~(Y ~ Z), model = ~lm_robust): The argument 'model = ' is
## deprecated. Please use '.method = ' instead.

## Warning in fn(data, ~(Y ~ Z), model = ~lm_robust): The argument 'model = ' is
## deprecated. Please use '.method = ' instead.

## Warning in fn(data, ~(Y ~ Z), model = ~lm_robust): The argument 'model = ' is
## deprecated. Please use '.method = ' instead.

## Warning in fn(data, ~(Y ~ Z), model = ~lm_robust): The argument 'model = ' is
## deprecated. Please use '.method = ' instead.

## Warning in fn(data, ~(Y ~ Z), model = ~lm_robust): The argument 'model = ' is
## deprecated. Please use '.method = ' instead.

## Warning in fn(data, ~(Y ~ Z), model = ~lm_robust): The argument 'model = ' is
## deprecated. Please use '.method = ' instead.

## Warning in fn(data, ~(Y ~ Z), model = ~lm_robust): The argument 'model = ' is
## deprecated. Please use '.method = ' instead.

## Warning in fn(data, ~(Y ~ Z), model = ~lm_robust): The argument 'model = ' is
## deprecated. Please use '.method = ' instead.

## Warning in fn(data, ~(Y ~ Z), model = ~lm_robust): The argument 'model = ' is
## deprecated. Please use '.method = ' instead.

## Warning in fn(data, ~(Y ~ Z), model = ~lm_robust): The argument 'model = ' is
## deprecated. Please use '.method = ' instead.

## Warning in fn(data, ~(Y ~ Z), model = ~lm_robust): The argument 'model = ' is
## deprecated. Please use '.method = ' instead.

## Warning in fn(data, ~(Y ~ Z), model = ~lm_robust): The argument 'model = ' is
## deprecated. Please use '.method = ' instead.

## Warning in fn(data, ~(Y ~ Z), model = ~lm_robust): The argument 'model = ' is
## deprecated. Please use '.method = ' instead.

## Warning in fn(data, ~(Y ~ Z), model = ~lm_robust): The argument 'model = ' is
## deprecated. Please use '.method = ' instead.

## Warning in fn(data, ~(Y ~ Z), model = ~lm_robust): The argument 'model = ' is
## deprecated. Please use '.method = ' instead.

## Warning in fn(data, ~(Y ~ Z), model = ~lm_robust): The argument 'model = ' is
## deprecated. Please use '.method = ' instead.

## Warning in fn(data, ~(Y ~ Z), model = ~lm_robust): The argument 'model = ' is
## deprecated. Please use '.method = ' instead.

## Warning in fn(data, ~(Y ~ Z), model = ~lm_robust): The argument 'model = ' is
## deprecated. Please use '.method = ' instead.

## Warning in fn(data, ~(Y ~ Z), model = ~lm_robust): The argument 'model = ' is
## deprecated. Please use '.method = ' instead.

## Warning in fn(data, ~(Y ~ Z), model = ~lm_robust): The argument 'model = ' is
## deprecated. Please use '.method = ' instead.

## Warning in fn(data, ~(Y ~ Z), model = ~lm_robust): The argument 'model = ' is
## deprecated. Please use '.method = ' instead.

## Warning in fn(data, ~(Y ~ Z), model = ~lm_robust): The argument 'model = ' is
## deprecated. Please use '.method = ' instead.

## Warning in fn(data, ~(Y ~ Z), model = ~lm_robust): The argument 'model = ' is
## deprecated. Please use '.method = ' instead.

## Warning in fn(data, ~(Y ~ Z), model = ~lm_robust): The argument 'model = ' is
## deprecated. Please use '.method = ' instead.

## Warning in fn(data, ~(Y ~ Z), model = ~lm_robust): The argument 'model = ' is
## deprecated. Please use '.method = ' instead.

## Warning in fn(data, ~(Y ~ Z), model = ~lm_robust): The argument 'model = ' is
## deprecated. Please use '.method = ' instead.

## Warning in fn(data, ~(Y ~ Z), model = ~lm_robust): The argument 'model = ' is
## deprecated. Please use '.method = ' instead.

## Warning in fn(data, ~(Y ~ Z), model = ~lm_robust): The argument 'model = ' is
## deprecated. Please use '.method = ' instead.

## Warning in fn(data, ~(Y ~ Z), model = ~lm_robust): The argument 'model = ' is
## deprecated. Please use '.method = ' instead.

## Warning in fn(data, ~(Y ~ Z), model = ~lm_robust): The argument 'model = ' is
## deprecated. Please use '.method = ' instead.

## Warning in fn(data, ~(Y ~ Z), model = ~lm_robust): The argument 'model = ' is
## deprecated. Please use '.method = ' instead.

## Warning in fn(data, ~(Y ~ Z), model = ~lm_robust): The argument 'model = ' is
## deprecated. Please use '.method = ' instead.

## Warning in fn(data, ~(Y ~ Z), model = ~lm_robust): The argument 'model = ' is
## deprecated. Please use '.method = ' instead.

## Warning in fn(data, ~(Y ~ Z), model = ~lm_robust): The argument 'model = ' is
## deprecated. Please use '.method = ' instead.

## Warning in fn(data, ~(Y ~ Z), model = ~lm_robust): The argument 'model = ' is
## deprecated. Please use '.method = ' instead.

## Warning in fn(data, ~(Y ~ Z), model = ~lm_robust): The argument 'model = ' is
## deprecated. Please use '.method = ' instead.

## Warning in fn(data, ~(Y ~ Z), model = ~lm_robust): The argument 'model = ' is
## deprecated. Please use '.method = ' instead.

## Warning in fn(data, ~(Y ~ Z), model = ~lm_robust): The argument 'model = ' is
## deprecated. Please use '.method = ' instead.

## Warning in fn(data, ~(Y ~ Z), model = ~lm_robust): The argument 'model = ' is
## deprecated. Please use '.method = ' instead.

## Warning in fn(data, ~(Y ~ Z), model = ~lm_robust): The argument 'model = ' is
## deprecated. Please use '.method = ' instead.

## Warning in fn(data, ~(Y ~ Z), model = ~lm_robust): The argument 'model = ' is
## deprecated. Please use '.method = ' instead.

## Warning in fn(data, ~(Y ~ Z), model = ~lm_robust): The argument 'model = ' is
## deprecated. Please use '.method = ' instead.

## Warning in fn(data, ~(Y ~ Z), model = ~lm_robust): The argument 'model = ' is
## deprecated. Please use '.method = ' instead.

## Warning in fn(data, ~(Y ~ Z), model = ~lm_robust): The argument 'model = ' is
## deprecated. Please use '.method = ' instead.

## Warning in fn(data, ~(Y ~ Z), model = ~lm_robust): The argument 'model = ' is
## deprecated. Please use '.method = ' instead.

## Warning in fn(data, ~(Y ~ Z), model = ~lm_robust): The argument 'model = ' is
## deprecated. Please use '.method = ' instead.

## Warning in fn(data, ~(Y ~ Z), model = ~lm_robust): The argument 'model = ' is
## deprecated. Please use '.method = ' instead.

## Warning in fn(data, ~(Y ~ Z), model = ~lm_robust): The argument 'model = ' is
## deprecated. Please use '.method = ' instead.

## Warning in fn(data, ~(Y ~ Z), model = ~lm_robust): The argument 'model = ' is
## deprecated. Please use '.method = ' instead.

## Warning in fn(data, ~(Y ~ Z), model = ~lm_robust): The argument 'model = ' is
## deprecated. Please use '.method = ' instead.

## Warning in fn(data, ~(Y ~ Z), model = ~lm_robust): The argument 'model = ' is
## deprecated. Please use '.method = ' instead.

## Warning in fn(data, ~(Y ~ Z), model = ~lm_robust): The argument 'model = ' is
## deprecated. Please use '.method = ' instead.

## Warning in fn(data, ~(Y ~ Z), model = ~lm_robust): The argument 'model = ' is
## deprecated. Please use '.method = ' instead.

## Warning in fn(data, ~(Y ~ Z), model = ~lm_robust): The argument 'model = ' is
## deprecated. Please use '.method = ' instead.

## Warning in fn(data, ~(Y ~ Z), model = ~lm_robust): The argument 'model = ' is
## deprecated. Please use '.method = ' instead.

## Warning in fn(data, ~(Y ~ Z), model = ~lm_robust): The argument 'model = ' is
## deprecated. Please use '.method = ' instead.

## Warning in fn(data, ~(Y ~ Z), model = ~lm_robust): The argument 'model = ' is
## deprecated. Please use '.method = ' instead.

## Warning in fn(data, ~(Y ~ Z), model = ~lm_robust): The argument 'model = ' is
## deprecated. Please use '.method = ' instead.

## Warning in fn(data, ~(Y ~ Z), model = ~lm_robust): The argument 'model = ' is
## deprecated. Please use '.method = ' instead.

## Warning in fn(data, ~(Y ~ Z), model = ~lm_robust): The argument 'model = ' is
## deprecated. Please use '.method = ' instead.

## Warning in fn(data, ~(Y ~ Z), model = ~lm_robust): The argument 'model = ' is
## deprecated. Please use '.method = ' instead.

## Warning in fn(data, ~(Y ~ Z), model = ~lm_robust): The argument 'model = ' is
## deprecated. Please use '.method = ' instead.

## Warning in fn(data, ~(Y ~ Z), model = ~lm_robust): The argument 'model = ' is
## deprecated. Please use '.method = ' instead.

## Warning in fn(data, ~(Y ~ Z), model = ~lm_robust): The argument 'model = ' is
## deprecated. Please use '.method = ' instead.

## Warning in fn(data, ~(Y ~ Z), model = ~lm_robust): The argument 'model = ' is
## deprecated. Please use '.method = ' instead.

## Warning in fn(data, ~(Y ~ Z), model = ~lm_robust): The argument 'model = ' is
## deprecated. Please use '.method = ' instead.

## Warning in fn(data, ~(Y ~ Z), model = ~lm_robust): The argument 'model = ' is
## deprecated. Please use '.method = ' instead.
\end{verbatim}

\begin{Shaded}
\begin{Highlighting}[]
\CommentTok{\# design0\_N100\_tau25\_sims has 200 rows}
\CommentTok{\# (2 tests * 100 random assignments)}
\CommentTok{\# just look at the first 6 rows}
\KeywordTok{head}\NormalTok{(design0\_N100\_tau25\_sims)}
\end{Highlighting}
\end{Shaded}

\begin{verbatim}
##                    design   N  tau sim_ID estimator term  estimate std.error
## 1 design0_N100_tau25_plus 100 0.25      1  t test 1    Z 0.1467999 0.2037624
## 2 design0_N100_tau25_plus 100 0.25      1  t test 2 <NA>        NA        NA
## 3 design0_N100_tau25_plus 100 0.25      2  t test 1    Z 0.4222026 0.2032860
## 4 design0_N100_tau25_plus 100 0.25      2  t test 2 <NA>        NA        NA
## 5 design0_N100_tau25_plus 100 0.25      3  t test 1    Z 0.3182903 0.2039122
## 6 design0_N100_tau25_plus 100 0.25      3  t test 2 <NA>        NA        NA
##   statistic    p.value    conf.low conf.high df outcome inquiry step_1_draw
## 1 0.7204464 0.47296517 -0.25755996 0.5511597 98       Y     ATE           1
## 2 0.7204464 0.47296531          NA        NA NA    <NA>    <NA>           1
## 3 2.0768897 0.04042793  0.01878815 0.8256171 98       Y     ATE           1
## 4 2.0768897 0.04053255          NA        NA NA    <NA>    <NA>           1
## 5 1.5609179 0.12176674 -0.08636693 0.7229475 98       Y     ATE           1
## 6 1.5609179 0.12179580          NA        NA NA    <NA>    <NA>           1
##   step_2_draw
## 1           1
## 2           1
## 3           2
## 4           2
## 5           3
## 6           3
\end{verbatim}

\begin{Shaded}
\begin{Highlighting}[]
\CommentTok{\# for each estimator, power =}
\CommentTok{\# proportion of simulations with}
\CommentTok{\# p.value \textless{} 0.5}
\NormalTok{design0\_N100\_tau25\_sims }\OperatorTok{\%\textgreater{}\%}
\StringTok{    }\KeywordTok{group\_by}\NormalTok{(estimator) }\OperatorTok{\%\textgreater{}\%}
\StringTok{    }\KeywordTok{summarize}\NormalTok{(}\DataTypeTok{pow =} \KeywordTok{mean}\NormalTok{(p.value }\OperatorTok{\textless{}}\StringTok{ }\FloatTok{0.05}\NormalTok{),}
        \DataTypeTok{.groups =} \StringTok{"drop"}\NormalTok{)}
\end{Highlighting}
\end{Shaded}

\begin{verbatim}
## # A tibble: 2 x 2
##   estimator   pow
##   <chr>     <dbl>
## 1 t test 1   0.21
## 2 t test 2   0.21
\end{verbatim}
\end{frame}

\hypertarget{power-with-covariate-adjustment}{%
\section{Power with covariate
adjustment}\label{power-with-covariate-adjustment}}

\begin{frame}{Covariate adjustment and power}
\protect\hypertarget{covariate-adjustment-and-power}{}
\begin{itemize}
\item
  Covariate adjustment can improve power because it mops up variation in
  the outcome variable.

  \begin{itemize}
  \item
    If prognostic, covariate adjustment can reduce variance
    dramatically. Lower variance means higher power.
  \item
    If non-prognostic, power gains are minimal.
  \end{itemize}
\item
  All covariates must be pre-treatment. Do not drop observations on
  account of missingness.

  \begin{itemize}
  \tightlist
  \item
    See the module on
    \href{threats-to-internal-validity-of-randomized-experiments.html}{threats
    to internal validity} and the
    \href{https://egap.org/resource/10-things-to-know-about-covariate-adjustment/}{10
    things to know about covariate adjustment}.
  \end{itemize}
\item
  Freedman's bias as n of observations decreases and K covariates
  increases.
\end{itemize}
\end{frame}

\hypertarget{power-with-covariate-adjustment-1}{%
\section{Power with covariate
adjustment}\label{power-with-covariate-adjustment-1}}

\begin{frame}{Covariate adjustment and power}
\protect\hypertarget{covariate-adjustment-and-power-1}{}
\begin{itemize}
\item
  Covariate adjustment can improve power because it mops up variation in
  the outcome variable.

  \begin{itemize}
  \item
    If prognostic, covariate adjustment can reduce variance
    dramatically. Lower variance means higher power.
  \item
    If non-prognostic, power gains are minimal.
  \end{itemize}
\item
  All covariates must be pre-treatment. Do not drop observations on
  account of missingness.

  \begin{itemize}
  \tightlist
  \item
    See the module on
    \href{threats-to-internal-validity-of-randomized-experiments.html}{threats
    to internal validity} and the
    \href{https://egap.org/resource/10-things-to-know-about-covariate-adjustment/}{10
    things to know about covariate adjustment}.
  \end{itemize}
\item
  Freedman's bias as n of observations decreases and K covariates
  increases.
\end{itemize}
\end{frame}

\begin{frame}{Blocking}
\protect\hypertarget{blocking}{}
\begin{itemize}
\item
  Blocking: randomly assign treatment within blocks

  \begin{itemize}
  \item
    ``Ex-ante'' covariate adjustment
  \item
    Higher precision/efficiency implies more power
  \item
    Reduce ``conditional bias'': association between treatment
    assignment and potential outcomes
  \item
    Benefits of blocking over covariate adjustment clearest in small
    experiments
  \end{itemize}
\end{itemize}
\end{frame}

\hypertarget{power-for-cluster-randomization}{%
\section{Power for cluster
randomization}\label{power-for-cluster-randomization}}

\begin{frame}{Power and clustered designs}
\protect\hypertarget{power-and-clustered-designs}{}
\begin{itemize}
\item
  Recall the \href{randomization.html}{randomization module}.
\item
  Given a fixed \(N\), a clustered design is weakly less powered than a
  non-clustered design.

  \begin{itemize}
  \tightlist
  \item
    The difference is often substantial.
  \end{itemize}
\item
  We have to estimate variance correctly:

  \begin{itemize}
  \tightlist
  \item
    Clustering standard errors (the usual)
  \item
    Randomization inference
  \end{itemize}
\item
  To increase power:

  \begin{itemize}
  \tightlist
  \item
    Better to increase number of clusters than number of units per
    cluster.
  \item
    How much clusters reduce power depends critically on the
    intra-cluster correlation (the ratio of variance within clusters to
    total variance).
  \end{itemize}
\end{itemize}
\end{frame}

\begin{frame}{A note on clustering in observational research}
\protect\hypertarget{a-note-on-clustering-in-observational-research}{}
\begin{itemize}
\item
  Often overlooked, leading to (possibly) wildly understated
  uncertainty.

  \begin{itemize}
  \item
    Frequentist inference based on ratio \(\hat{\beta}/\hat{se}\)
  \item
    If we underestimate \(\hat{se}\), we are much more likely to reject
    \(H_0\). (Type-I error rate is too high.)
  \end{itemize}
\item
  Many observational designs much less powered than we think they are.
\end{itemize}
\end{frame}

\begin{frame}{EGAP Power Calculator}
\protect\hypertarget{egap-power-calculator}{}
\begin{itemize}
\item
  Try the calculator at: \url{https://egap.shinyapps.io/power-app/}
\item
  For cluster randomization designs, try adjusting:

  \begin{itemize}
  \tightlist
  \item
    Number of clusters
  \item
    Number of units per clusters
  \item
    Intra-cluster correlation
  \item
    Treatment effect
  \end{itemize}
\end{itemize}
\end{frame}

\begin{frame}{Comments}
\protect\hypertarget{comments}{}
\begin{itemize}
\item
  Know your outcome variable.
\item
  What effects can you realistically expect from your treatment?
\item
  What is the plausible range of variation of the outcome variable?

  \begin{itemize}
  \tightlist
  \item
    A design with limited possible movement in the outcome variable may
    not be well-powered.
  \end{itemize}
\end{itemize}
\end{frame}

\begin{frame}{Conclusion: How to improve your power}
\protect\hypertarget{conclusion-how-to-improve-your-power}{}
\begin{enumerate}
\item
  Increase the \(N\)

  \begin{itemize}
  \tightlist
  \item
    If clustered, increase the number of clusters if at all possible
  \end{itemize}
\item
  Strengthen the treatment
\item
  Improve precision

  \begin{itemize}
  \item
    Covariate adjustment
  \item
    Blocking
  \end{itemize}
\item
  Better measurement of the outcome variable
\end{enumerate}
\end{frame}

\end{document}
